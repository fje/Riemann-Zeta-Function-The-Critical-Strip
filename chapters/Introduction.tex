The Riemann zeta function has many interesting properties in the critical strip that have a huge impact on the analytic number theory, especially concerning the distribution of prime numbers. We want to have a closer look at its zero-free regions and find good estimates of the function's growth under certain circumstances. These will be needed later on in order to prove the prime number theorem and to show that the Riemann zeta function has infinitely many zeros on the critical line. \\
This essay is a continuation of \cite{Jesacher2016} and is mainly based on \cite{Stein2003} and \cite{Titchmarsh1986}. \\
Henceforth let $s \in \field{C}$ be a complex number where $\sigma \coloneqq \Re(s)$ and $t \coloneqq \Im(s)$ and if nothing else is stated let $p \in \field{N}$ be a prime number.


\section{Properties of $\zeta$ in the critical strip}
First of all we want to draw some conclusions about the function's behaviour at the borders of the critical strip and show the absence of zeros on the lines $\sigma = 0$ and $\sigma = 1$.


\begin{lemma}\label{lem:LogEulerProd}
	If $\sigma > 1$ then
\begin{equation*}
	\log(\zeta(s)) = \sum _{p,m} \frac{p^{-ms}}{m}.
\end{equation*}
\end{lemma}
\begin{proof}
	Let $x \in \cbr{0, 1}$. Using the power series expansion for the logarithm
\begin{equation*}
	\log\cbr{\frac{1}{1 - x}} = -\log(1 + (-x)) = \sum _{m = 1} ^\infty \cbr{\frac{x^m}{m}}.
\end{equation*}
	and applying this to $\zeta$'s absolutely convergent Euler product representation while assuming that $s \in \cbr{0 , \infty}$ yields
\begin{equation*}
\begin{aligned}	
	\log(\zeta(s)) 
		&= \log\cbr{\prod _p \frac{1}{1 - p^{-s}}} \\
		&= \sum _p \log\cbr{\frac{1}{1 - p^{-s}}}
		= \sum _{p,m} \frac{p^{-ms}}{m} = \sum _{n=1} ^{\infty} c_n n^{-s},
\end{aligned}
\end{equation*}
	where
\begin{equation*}
    c_n =
    \left\{
    	\begin{array}{ll}
        	\frac{1}{m}, & \text{for } n = p^m, \\
        	0, & \text{otherwise.}
        \end{array}
	\right.
\end{equation*}
	By analytic continuation this also holds for $s \in \field{C}$ with $\sigma > 1$. In particular if $U = \fbr{s \in \field{C} \colon \sigma > 1}$ then $\zeta \in \mathcal{H}(U)$ where $U$ is simply connected and $\zeta(s) \neq 0$ for all $s \in U$. Hence there exists a $g \in \mathcal{H}(U)$ such that $\zeta(s) = e^{g(s)}$. Therefore $\log(\zeta(s))$ is well defined for all $s \in U$.
\end{proof}


\begin{lemma}\label{lem:CosAddTh}
	If $\theta \in \field{R}$ then
\begin{equation*}
	3 + 4 \cos(\theta) + \cos(2 \theta) \geq 0.
\end{equation*}
\end{lemma}
\begin{proof}
	The application of the well known trigonometric identity
\begin{equation*}
	\cos(\theta)^2 = \frac{1 + \cos(2 \theta)}{2}
\end{equation*}
	immediately leads to
\begin{equation*}
	3 + 4 \cos(\theta) + \cos(2 \theta) = 2 + 4 \cos(\theta) + 2 \cos(\theta)^2 = 2(1 + \cos(\theta))^2 \geq 0.
\end{equation*}
\end{proof}


\begin{corollary}\label{cor:LogZeta}
	If $\sigma > 1$ then
\begin{equation*}
	\log\cbr{\abs{\zeta^3(\sigma) \zeta^4(\sigma + it) \zeta(\sigma + 2it)}} \geq 0.
\end{equation*}
\end{corollary}
\begin{proof}
	We write
\begin{equation*}
\begin{aligned}	
	\Re(n^{-s}) 
		&= \Re(e^{-(\sigma + it) \log(n)}) \\
		&= e^{-\sigma \log(n)} \cos(t \log(n)) 
		= n^{-\sigma} \cos(t \log(n))
\end{aligned}
\end{equation*}
	and for $r > 0$ we have
\begin{equation*}
	\Re(\log(r e^{i \theta})) = \log(r) = \log\cbr{\abs{r e^{i\theta}}}.
\end{equation*}
	Therefore Lemma~\ref{lem:LogEulerProd} and Lemma~\ref{lem:CosAddTh} directly give us our final result
\begin{equation*}
\begin{aligned}
	\log\cbr{\abs{\zeta^3(\sigma) \zeta^4(\sigma + it) \zeta(\sigma + 2it)}} &= \\
	3\log(\abs{\zeta(\sigma)}) + 4\log(\abs{\zeta(\sigma + it)}) + \log(\abs{\zeta(\sigma + 2it)}) &= \\
	3\Re(\log(\zeta(\sigma))) + 4\Re(\log(\zeta(\sigma + it))) + \Re(\log(\zeta(\sigma + 2it))) &= \\
	\sum _{n=1} ^{\infty} c_n n^{-\sigma}(3 + 4\cos(t \log(n)) + \cos(2 t \log(n))) &\geq 0.
\end{aligned}
\end{equation*}
\end{proof}


\begin{theorem}
	It holds that $\zeta(1 + it) \neq 0$ for all $t \in \field{R}$.
\end{theorem}
\begin{proof}
	Let us suppose there exists a constant $t_0 \neq 0$ such that $\zeta(1 + t_0) = 0$. We know that $\zeta$ is holomorphic on $\field{C} \mysetminus \{1\}$, particularly in $s = 1 + it_0$. Hence
\begin{equation*}
	\lim\limits_{h \to 0} \frac{\zeta(1 + it_0 + h) - \zeta(1 + it_0)}{h} = \lim\limits_{\sigma \to 1} \frac{\zeta(\sigma + it_0)}{\sigma - 1} \in \field{C}.
\end{equation*}
	Thus
\begin{equation*}
	\abs{\zeta^4(\sigma + it_0)} = \mathcal{O}\cbr{\abs{\sigma - 1}^4} \text{ as } \sigma \to 1.
\end{equation*}
	Since $\zeta$ has a simple pole in $s = 1$ we get
\begin{equation*}
	\lim\limits_{\sigma \to 1}\zeta(\sigma) (\sigma - 1) \in \field{C}.
\end{equation*}
	Thus we find that
\begin{equation*}
	 \abs{\zeta^3(\sigma)} = \mathcal{O}\cbr{\abs{\sigma - 1}^{-3}} \text{ as } \sigma \to 1.
\end{equation*}
	Since $\zeta$ is also holomorphic in $s = 1 + 2it_0$ the function is locally bounded there, therefore we have
\begin{equation*}
	 \abs{\zeta(\sigma + 2it_0)} = \mathcal{O}\cbr{1} \text{ as } \sigma \to 1.
\end{equation*}
	Hence we get
\begin{equation*}
	 \lim\limits_{\sigma \to 1} \abs{\zeta(\sigma)^3 \zeta(\sigma + it_0)^4 \zeta(\sigma + 2it_0)} = 0.
\end{equation*}
	But this contradicts Corollary~\ref{cor:LogZeta} since $\log(x) < 0$ for all $x \in \cbr{0 , 1}$.
\end{proof}


\begin{corollary}
	It holds that $\zeta(it) \neq 0$ for all $t \in \field{R}$.
\end{corollary}
\begin{proof}
	This proof relies on the fact that the $\xi$ function (see Definition~\ref{def:XiDefinition}) satisfies the symmetry $\xi(s) = \xi(1 - s)$ (see Proposition~\ref{pro:XiIdentity}). This, and the fact that only $\zeta$ can produce zeros in $\xi$, immediately yields
\begin{equation*}
	\zeta(1 + it) \neq 0 \implies \xi(1 + it) = \xi(-it) \neq 0 \implies \zeta(-it) \neq 0,	
\end{equation*}
	for all $t \in \field{R}$.
\end{proof}


\section{Estimates involving $\zeta$}
We want to study the growth of $\zeta$, $\zeta'$ and $\zeta^{-1}$ in the half-plane $\sigma > 0$ in more detail in order to find good estimates for later applications.

\begin{proposition}
	There exists a sequence of entire functions $\fbr{\delta_n(s)} _{n = 1} ^\infty$ that satisfy the estimate $\abs{\delta_n(s)} \leq \frac{\abs{s}}{n^{\sigma + 1}}$ and such that
\begin{equation}\label{equ:SumDelta}
	\sum _{1 \leq n < N} \frac{1}{n^s} - \int _1 ^N \frac{dx}{x^s} = \sum _{1 \leq n < N} \delta_n(s),
\end{equation}
	whenever $N$ is an integer $> 1$ and $\sigma > 0$.
\end{proposition}
\begin{proof}
	In order to prove our claim we compare $\sum _{1 \leq n < N} n^{-s}$ with $\sum _{1 \leq n < N} \int _n ^{n + 1} x^{-s} dx$, and set
\begin{equation}\label{equ:DeltaN}
	\delta_n(s) = \int _n ^{n + 1} \cbr{\frac{1}{n^s} - \frac{1}{x^s}} dx.
\end{equation}
	Let $f(x) = x^{-s}$. We have
\begin{equation*}
\begin{aligned}	
	\abs{\frac{1}{n^s} - \frac{1}{x^s}} = \abs{\int _n ^x f'(y) dy} 
	&\leq \int _n ^x \abs{f'(y)} dy \\
	&\leq \sup _{y \in \bbr{n, x}} \abs{f'(y)} = \abs{f'(n)} = \frac{\abs{s}}{n^{\sigma + 1}},
\end{aligned}
\end{equation*}
	whenever $x \in \bbr{n, n + 1}$. Therefore $\abs{\delta_n(s)} \leq \frac{\abs{s}}{n^{\sigma + 1}}$, and since
\begin{equation*}
	\int _1 ^N \frac{dx}{x^s} = \sum _{1 \leq n < N} \int _n ^{n + 1} \frac{dx}{x^s},
\end{equation*}
	our proof is complete.
\end{proof}


\begin{corollary}\label{cor:ZetaCont}
	For $\sigma > 0$ we have
\begin{equation*}
	\zeta(s) - \frac{1}{s - 1} = H(s),
\end{equation*}
	where $H(s) = \sum _{n = 1} ^\infty \delta_n(s)$ is holomorphic in the half-plane $\sigma > 0$. 
\end{corollary}
\begin{proof}
	We first assume that $\sigma > 1$. When $N \to \infty$ in formula~(\ref{equ:SumDelta}) we observe that by the estimate $\abs{\delta_n(s)} \leq \frac{\abs{s}}{n^{\sigma + 1}}$ we have uniform convergence of the series $\sum _{n = 1} ^\infty \delta_n(s)$ on compact sets in any half plane $\sigma \geq \delta$ when $\delta > 0$. Since $\sigma > 1$ the series $\sum _{n = 1} ^{\infty} n^{-s}$ converges to $\zeta(s)$. This proves our assertion when $\sigma > 1$. The uniform convergence shows that $\sum _{n = 1} ^\infty \delta_n(s)$ is holomorphic when $\sigma > 0$ and thus shows that $\zeta$ is extendable to that half-plane and that the identity continues to hold there.
\end{proof}


\begin{proposition}
	For each $\sigma_0 \in \bbr{0 , 1}$ and every $\eps > 0$, there exists a constant $c_\eps$, such that
\begin{equation}\label{equ:ZetaEst1}
	\abs{\zeta(s)} \leq c_\eps \abs{t}^{1 - \sigma_0 + \eps}, \textit{ if } \sigma_0 \leq \sigma \textit{ and } \abs{t} \geq 1
\end{equation}
	and
\begin{equation}\label{equ:ZetaEst2}
	\abs{\zeta'(s)} \leq c_\eps \abs{t}^\eps, \textit{ if } 1 \leq \sigma \textit{ and } \abs{t} \geq 1.
\end{equation}
\end{proposition}
\begin{proof}
	For the proof we recall the estimate $\abs{\delta_n(s)} \leq \frac{\abs{s}}{n^{\sigma + 1}}$. We also have the estimate $\abs{\delta_n(s)} \leq \frac{2}{n^\sigma}$, which follows from the expression for $\delta_n$ given by~(\ref{equ:DeltaN}) and the fact that $\abs{n^{-s}} = n^{-\sigma}$ and $\abs{x^{-s}} \leq n^{-\sigma}$ if $x \geq n$. We then combine these two estimates for $\abs{\delta_n(s)}$ via the observation that $A = A^\delta A^{1 - \delta}$, to obtain the bound
\begin{equation*}
	\abs{\delta_n(s)} \leq \cbr{\frac{\abs{s}}{n^{\sigma_0 + 1}}}^\delta \cbr{\frac{2}{n^{\sigma_0}}}^{1 - \delta} \leq \frac{2 \abs{s}^\delta}{n^{\sigma_0 + \delta}},
\end{equation*}
	as long as $\delta \geq 0$. Now choose $\delta = 1 - \sigma_0 + \eps$ and apply the estimate to Corollary~\ref{cor:ZetaCont}. Then, with $\sigma \geq \sigma_0$, we find
\begin{equation*}
	\abs{\zeta(s)} \leq \abs{\frac{1}{s - 1}} + 2 \abs{s}^{1 - \sigma_0 + \eps} \sum _{n = 1} ^\infty \frac{1}{n^{1 + \eps}},
\end{equation*}
	and conclusion~(\ref{equ:ZetaEst1}) is proved. By the Cauchy integral formula,
\begin{equation*}
	\zeta'(s) = \frac{1}{2 \pi r} \int _0 ^{2 \pi} \zeta(s + re^{i \theta}) e^{-i \theta} d\theta,
\end{equation*}
	where the integration is taken over a circle of radius $r$ centred at $s$. Now choose $r = \eps$ and observe that this circle lies in the half-plane $\sigma \geq 1 - \eps$, and so the estimate~(\ref{equ:ZetaEst2}) follows as a consequence of the estimate~(\ref{equ:ZetaEst1}) on replacing $2 \eps$ by $\eps$.
\end{proof}


\begin{proposition}
	For every $\eps > 0$ there exists a constant $c_\eps$, such that $\frac{1}{\abs{\zeta(s)}} \leq c_\eps \abs{t}^\eps$, if $\sigma \geq 1$ and $\abs{t} \geq 1$.
\end{proposition}
\begin{proof}
	By Corollary~\ref{cor:LogZeta} we have that
\begin{equation*}
	\abs{\zeta^3(\sigma) \zeta^4(\sigma + it) \zeta(\sigma + 2it)} \geq 1,
\end{equation*}
	whenever $\sigma > 1$. Using our previous estimate~(\ref{equ:ZetaEst1}) for $\zeta$ we find that
\begin{equation*}
	\abs{\zeta^4(\sigma + it)} \geq c \abs{\zeta^{-3}(\sigma)} \abs{t}^{-\eps} \geq c' (\sigma - 1)^3 \abs{t}^{-\eps}
\end{equation*}
	for all $\sigma \geq 1$ and $\abs{t} \geq 1$. Thus
\begin{equation*}
	\abs{\zeta(\sigma + it)} \geq c' (\sigma - 1)^{\frac{3}{4}} \abs{t}^{-\frac{\eps}{4}}.
\end{equation*}
	We now consider two cases, depending on whether $\sigma - 1 \geq A\abs{t}^{-5 \eps}$ holds for an appropriate constant $A \in \field{R}_{> 0}$ or not. If this inequality does hold it follows
\begin{equation*}
	\abs{\zeta(\sigma + it)} \geq A' \abs{t}^{-4 \eps}.
\end{equation*}
	By replacing $4 \eps$ by $\eps$ we can conclude the proof in this case. If, however, $\sigma - 1 < A\abs{t}^{-5 \eps}$ then we first choose $\sigma' > \sigma$ with $\sigma' - 1 = A \abs{t}^{-5 \eps}$. The triangle inequality then implies
\begin{equation*}
	\abs{\zeta(\sigma + it)} \geq \abs{\zeta(\sigma' + it)} - \abs{\zeta(\sigma' + it) - \zeta(\sigma + it)},
\end{equation*}
	and combined with the previous estimate~(\ref{equ:ZetaEst2}) for $\zeta'$ we obtain
\begin{equation*}
\begin{aligned}	
	\abs{\zeta(\sigma' + it) - \zeta(\sigma + it)} = \abs{\int _\sigma ^{\sigma'} \zeta'(x + it) dx}
	&\leq \int _\sigma ^{\sigma'} \abs{\zeta'(x + it)} dx \\ 
	&\leq c'' (\sigma' - \sigma) \abs{t}^{\eps} \\
	&\leq c''(\sigma' - 1)\abs{t}^{\eps}.
\end{aligned}
\end{equation*}
	These observations, together with our result above where we set $\sigma = \sigma'$, show that
\begin{equation*}
	\abs{\zeta(\sigma + it)} \geq c'(\sigma' - 1)^{\frac{3}{4}}\abs{t}^{-\frac{\eps}{4}} - c''(\sigma' - 1)\abs{t}^{\eps}.
\end{equation*}
	We can now choose $A = \cbr{\frac{c'}{2 c''}}^4$, and recall that $\sigma' - 1 = A \abs{t}^{-5 \eps}$. This gives precisely
\begin{equation*}
	c'(\sigma' - 1)^{\frac{3}{4}} \abs{t}^{-\frac{\eps}{4}} = 2 c''(\sigma' - 1) \abs{t}^{\eps}
\end{equation*}
	and therefore
\begin{equation*}
	\abs{\zeta(\sigma + it)} \geq A''\abs{t}^{-4 \eps}.
\end{equation*}
	By replacing $4 \eps$ by $\eps$ the desired inequality is established and therefore the proof is complete.
\end{proof}


\begin{corollary}\label{cor:ZetaQuotEst}
	For every $\eps > 0$ we have $\abs{\frac{\zeta'(s)}{\zeta(s)}} \leq c_\eps \abs{t}^\eps$ for $c_\eps \in \field{R}_{> 0}$ if $\sigma \geq 1$ and $\abs{t} \geq 1$.
\end{corollary}
\begin{proof}
	Our desired result immediately follows from combining the previous estimates for $\zeta'$ and $\frac{1}{\zeta}$.
\end{proof}


\section{Mellin Transform}


\begin{definition}
	The function defined by
\begin{equation*}
	\fbr{\mathcal{M}f}(s) = \int _0 ^\infty x^{s - 1} f(x) dx.
\end{equation*}
	is called the Mellin transform of the function $f$. Conversely
\begin{equation*}
	\fbr{\mathcal{M}^{-1} \varphi}(x) = \frac{1}{2 \pi i} \int _{c - i \infty} ^{c + i \infty} x^{-s} \varphi(s) ds.
\end{equation*}
	defines the inverse Mellin transform.
\end{definition}


\begin{theorem}[Mellin's Inversion Formula]
	Let $\varphi(s)$ be analytic for $\sigma \in \cbr{a, b}$ and $\varphi(s) \to 0$ uniformly as $t \to \pm\infty$ with its integral along the line $\sigma = c$ converging absolutely where $c \in \cbr{a, b}$. Then if
\begin{equation*}
	f(x) = \fbr{\mathcal{M}^{-1} \varphi}(x) = \frac{1}{2 \pi i} \int _{c - i \infty} ^{c + i\infty} x^{-s} \varphi(s) ds
\end{equation*}
	we have that
\begin{equation*}
	\varphi(s) = \fbr{\mathcal{M} f}(s) = \int _0 ^\infty x^{s - 1} f(x) dx.
\end{equation*}
	Conversely suppose $f$ is piecewise continuous on the positive real numbers and suppose the integral
\begin{equation*}
	\varphi(s) = \int _0 ^\infty x^{s - 1} f(x) dx
\end{equation*}
	is absolutely convergent for $\sigma \in \cbr{a, b}$. Then $f$ is recoverable via the inverse Mellin transform from its Mellin transform.
\end{theorem}
\begin{proof}
	This theorem follows from the inversion formula for the Laplace transform and the proof can, for example, be found in $\cite{McLachlan1953}$.
\end{proof}


\section{The Poisson Summation Formula}


\begin{definition}
	For each $a > 0$ we define $\mathcal{F}_a$ to be the space of all functions $f$ that satisfy the following two conditions:
\begin{enumerate}
	\item[(i)] The function $f$ is holomorphic on the horizontal strip
		\begin{equation*}
			S_a = \fbr{s \in \field{C} \colon \abs{t} < a}.
		\end{equation*}
	\item[(ii)] There exists a constant $A > 0$ such that
		\begin{equation*}
			\abs{f(s)} \leq \frac{A}{1 + \sigma^2} \hsp \textit{ for all } \sigma \in \field{R}, \abs{t} < a.
		\end{equation*}
\end{enumerate}
\end{definition}


\begin{theorem}[Poisson Summation Formula]
	For $f \in \mathcal{F}_a$ for some $a > 0$, the Poisson summation may be stated as
\begin{equation*}
	\sum _{n = -\infty} ^\infty f(n) = \sum _{k = -\infty} ^\infty \hat{f}(k),
\end{equation*}
	where $\hat{f}$ is the Fourier transform of $f \big\vert _\field{R}$.
\end{theorem}
\begin{proof}
	The entire proof can be found in \cite{Stein2003}.
\end{proof}