In this chapter we will introduce the theta function, define the xi function and prove some important identities in order to come up with an integral formula that allows us to draw a far-reaching conclusion about the occurrence of zeros of the Riemann zeta function on the critical line.


\section{The Theta Function}


\begin{definition}
	The function
\begin{equation*}
\begin{aligned}
	&\vartheta \colon \field{R}_{> 0} \to \field{R}, \\
	&\vartheta(x) = \sum _{n = -\infty} ^{\infty} e^{-\pi n^2 x}
\end{aligned}
\end{equation*}
	is known as a version of the renowned "theta" function. For further references we also introduce two auxiliary functions
\begin{equation*}
	\omega(x) = \frac{\vartheta(x) - 1}{2} = \sum _{n = 1} ^{\infty} e^{-\pi n^2 x}
\end{equation*}
	and
\begin{equation*}
	\theta(x) = \omega(x) - \frac{1}{2 \sqrt{x}} = \sum _{n = 1} ^{\infty} e^{-\pi n^2 x} - \frac{1}{2 \sqrt{x}}.
\end{equation*}
\end{definition}


\begin{remark}
	It is known that the well-behaved monotonically increasing function $\theta$ is defined for $0 \leq x < \infty$.
\end{remark}


\begin{proposition}
	If $f(x) = e^{-\pi x^2}$ then $f(x) = \hat{f}(\xi)$, where $\hat{f}$ is the Fourier transform of $f$, or more formally
\begin{equation}\label{equ:FourierSame}
		\int _{-\infty} ^\infty e^{-\pi x^2} e^{-2 \pi i x \xi} dx = e^{-\pi \xi^2}.
\end{equation}
\end{proposition}
\begin{proof}
\begin{figure}[!htb]
\begin{minipage}[c]{0.5\textwidth}
\centering
\begin{equation*}
\begin{aligned}
	&\begin{aligned}
		\gamma_{1_+} \colon \bbr{-R, R} &\to \field{C}, \\ r &\mapsto r,
	\end{aligned} \\
	&\begin{aligned}
		\gamma_{1_-} \colon \bbr{-R, R} &\to \field{C}, \\ r &\mapsto -r + i\xi,
	\end{aligned} \\
	&\begin{aligned}
		\gamma_{2_+} \colon \bbr{0, \xi} &\to \field{C}, \\ t &\mapsto R + it,
	\end{aligned} \\
	&\begin{aligned}
		\gamma_{2_-} \colon \bbr{0, \xi} &\to \field{C}, \\ t &\mapsto -R + i(\xi - t)
	\end{aligned}
\end{aligned}
\end{equation*}
\end{minipage}
\begin{minipage}[c]{0.5\textwidth}
\raggedleft
\begin{tikzpicture}
	[decoration={
		markings,
		mark=at position 1cm with {\arrow[line width=1pt]{>}},
		mark=at position 3cm with {\arrow[line width=1pt]{>}},
		mark=at position 4.5cm with {\arrow[line width=1pt]{>}},
		mark=at position 6cm with {\arrow[line width=1pt]{>}},
		mark=at position 8cm with {\arrow[line width=1pt]{>}},
		mark=at position 9.5cm with {\arrow[line width=1pt]{>}},
	}]
	% The axes
	\draw[help lines, very thin, ->] (-3,0) -- (3,0) coordinate (xaxis);
	\draw[help lines, very thin, ->] (0,-2) -- (0,2) coordinate (yaxis);
	\node[below, gray] at (xaxis) {$\Re(z)$};
	\node[left, gray] at (yaxis) {$\Im(z)$};

	% The path
	\path[draw,line width=0.6pt,postaction=decorate] (-2, 0) -- (2, 0) -- (2, 1) -- (-2, 1) -- (-2, 0);
	
	
	% The labels

	\node[below right] {$0$};

	\draw[xshift=0cm] (0,0) node[circle,fill,inner sep=1.2pt](a){};
	\draw[xshift=0cm] (-2,0) node[circle,fill,inner sep=1.2pt](a){};
	\draw[xshift=0cm] (2,0) node[circle,fill,inner sep=1.2pt](a){};
	\draw[xshift=0cm] (2,1) node[circle,fill,inner sep=1.2pt](a){};
	\draw[xshift=0cm] (-2,1) node[circle,fill,inner sep=1.2pt](a){};

	\node at (-0.5, -0.4) {$\gamma_{1_+}$};
	\node at (2.5, 0.5) {$\gamma_{2_+}$};
	\node at (0.5, 1.3) {$\gamma_{1_-}$};
	\node at (-2.5, 0.5) {$\gamma_{2_-}$};
	
	\node at (-2.3, -0.4) {$-R$};
	\node at (2.1, -0.4) {$R$};
	\node at (2.5, 1.4) {$R + i\xi$};
	\node at (-2.6, 1.4) {$-R + i\xi$};
\end{tikzpicture}
\end{minipage}
\caption{Contour $\gamma$}
\label{fig:ContourC}
\end{figure}
	If $\xi = 0$ we already know very well that
\begin{equation*}
	\int _{-\infty} ^\infty e^{-\pi x^2} dx = 1.
\end{equation*}
	Now, let us suppose $\xi > 0$ and consider the function $f(z) = e^{-\pi z^2}$ for $z \in \field{C}$. $f$ is clearly entire. According to Cauchy's theorem we have
\begin{equation*}
	\int _{\gamma} f(z) dz = 0,
\end{equation*}
	where $\gamma$ is the contour in Figure~\ref{fig:ContourC}. For the integral over $\gamma_{1_+}$ we simply have
\begin{equation*}
	\lim _{R \to \infty} \int _{-R} ^R e^{-\pi x^2} dx = 1.
\end{equation*}
	For the integral over $\gamma_{2_+}$ for a fixed $\xi$ we have
\begin{equation*}
\begin{aligned}	
	I(R) 
	&\coloneqq \int _0 ^\xi f(R + i y) i dy \\
	&= \int _0 ^\xi e^{-\pi (R^2 + 2iRy - y^2)} i dy \to 0
\end{aligned}
\end{equation*}
	as $R \to \infty$, since we may estimate it by
\begin{equation*}
	\abs{I(R)} \leq Ce^{-\pi R^2}.
\end{equation*}
	The integral for $\gamma_{2_-}$ also goes to zero for the same reasons. Finally, the remaining integral over $\gamma_{1_-}$ is
\begin{equation*}
	\int _{R} ^{-R} e^{-\pi (x + i \xi)^2} dx = -e^{\pi \xi^2} \int _{-R} ^R e^{-\pi x^2} e^{-2 \pi i x \xi} dx.
\end{equation*}
	Thus for $R \to \infty$ the limit gives
\begin{equation*}
	1 - e^{\pi \xi^2} \int _{-\infty} ^{\infty} e^{-\pi x^2} e^{-2 \pi i x \xi} dx = 0
\end{equation*}
	and our desired formula is established. In the case $\xi < 0$, we then consider the rectangle in the lower half plane.
\end{proof}


\begin{lemma}
	For $x > 0$ we have the identity
\begin{equation*}
	\vartheta(x) = \frac{1}{\sqrt{x}} \vartheta\cbr{\frac{1}{x}}.
\end{equation*}
\end{lemma}
\begin{proof}
	The change of variables $x \to \sqrt{y} x$ in the integral formula~(\ref{equ:FourierSame}) shows that the Fourier transform of the function $f(x) = e^{-\pi y x^2}$ is in fact $\hat{f}(\xi) = \frac{1}{\sqrt{y}} e^{-\frac{\pi \xi^2}{y}}$. Apparently $f(x)$ belongs to $\mathcal{F}_a$. Thus we may apply the Poisson summation formula~\ref{the:Poisson} to the pair $\cbr{f, \hat{f}}$ and we get
\begin{equation*}
	\vartheta(x) = \sum _{n = -\infty} ^\infty e^{-\pi x n^2} = \sum _{n = -\infty} ^\infty \frac{1}{\sqrt{x}} e^{-\frac{\pi \xi^2}{x}} = \frac{1}{\sqrt{x}} \vartheta\cbr{\frac{1}{x}}.
\end{equation*}
\end{proof}


\begin{corollary}\label{cor:ThetaIdentity}
	We have the identities
\begin{equation*}
	\omega(x) = \frac{1}{\sqrt{x}} \omega\cbr{\frac{1}{x}} + \frac{1}{2\sqrt{x}} - \frac{1}{2}.
\end{equation*}
	and
\begin{equation*}
	\theta(x) = \frac{1}{\sqrt{x}} \theta\cbr{\frac{1}{x}}.
\end{equation*}
\end{corollary}
\begin{proof}
	For the first identity we have
\begin{equation*}
	2 \omega(x) + 1 = \vartheta(x) = \frac{1}{\sqrt{x}} \vartheta\cbr{\frac{1}{x}} = \frac{1}{\sqrt{x}} \cbr{2 \omega\cbr{\frac{1}{x}} + 1}
\end{equation*}
	and just solve the equation for $\omega(x)$. The second identity follows directly from the first by writing
\begin{equation*}
\begin{aligned}	
	\theta(x) = \omega(x) - \frac{1}{2 \sqrt{x}} 
	&= \frac{1}{\sqrt{x}} \omega\cbr{\frac{1}{x}} - \frac{1}{2} \\ 
	&= \frac{1}{\sqrt{x}} \cbr{\omega\cbr{\frac{1}{x}} - \frac{\sqrt{x}}{2}} = \frac{1}{\sqrt{x}} \theta\cbr{\frac{1}{x}}.
\end{aligned}
\end{equation*}
\end{proof}


\section{The Xi Function}


\begin{definition}\label{def:XiDefinition}
	Let
\begin{equation*}
\begin{aligned}
	&\xi \colon \field{C} \to \field{C}, \\
	&\xi(s) = \frac{s(s - 1)}{2} \pi ^{-\frac{s}{2}} \Gamma\cbr{\frac{s}{2}} \zeta(s)
\end{aligned}
\end{equation*}
	be the xi function and for further reference we define
\begin{equation*}
	\xi^*(s) = \frac{2}{s(s - 1)} \xi(s) = \pi ^{-\frac{s}{2}} \Gamma\cbr{\frac{s}{2}} \zeta(s)
\end{equation*}
\end{definition}


\begin{remark}
	$\xi$ is entire since $s$ and $s - 1$ cancel out the simple poles of $\Gamma$ in $s = 0$ and $\zeta$ in $s = 1$ and since the trivial zeros of $\zeta$ eradicate the simple poles of $\Gamma$ in $s = -2n$ for $n \in \field{N}_{> 0}$.
\end{remark}


\begin{proposition}\label{pro:XiIdentity}
	We have the identity
\begin{equation*}
	\xi(s) = \xi(1 - s).
\end{equation*}
\end{proposition}
\begin{proof}
	Let us recall the formula
\begin{equation*}
	\Gamma\cbr{\frac{s}{2}} = \int _0 ^\infty e^{-y} y^{\frac{s}{2} - 1} dy,
\end{equation*}	
	whenever $\sigma > 0$. We observe that if $n \geq 1$ and by substituting $y = x \pi n^2$ we may write
\begin{equation*}
	\int _0 ^\infty e^{-n^2 \pi x} x^{\frac{s}{2} - 1} dx = n^{-s} \pi^{-\frac{s}{2}}\Gamma\cbr{\frac{s}{2}}.
\end{equation*}
	Hence if $\sigma > 1$ we get
\begin{equation*}
\begin{aligned}
	\xi^*(s) = \pi^{-\frac{s}{2}} \Gamma\cbr{\frac{s}{2}} \zeta(s) 
		&= \sum _{n = 1} ^\infty \int _0 ^\infty x^{\frac{s}{2} - 1} e^{-n^2 \pi x} dx \\
		&= \int _0 ^\infty x^{\frac{s}{2} - 1} \sum _{n = 1} ^\infty e^{-n^2 \pi x} dx
		= \int _0 ^\infty x^{\frac{s}{2} - 1} \omega(x) dx,
\end{aligned}
\end{equation*}
	where we can interchange sum and integral because of absolute convergence. We now use Corollary~\ref{cor:ThetaIdentity} and may now write
\begin{equation*}
\begin{aligned}	
	\xi^*(s) 
	&= \\
		\int _0 ^1 x^{\frac{s}{2} - 1} \omega(x) dx  + \int _1 ^\infty x^{\frac{s}{2} - 1} \omega(x) dx &= \\
		\int _0 ^1 x^{\frac{s}{2} - 1} \cbr{\frac{1}{\sqrt{x}} \omega\cbr{\frac{1}{x}} + \frac{1}{2\sqrt{x}} - \frac{1}{2}} dx  + \int _1 ^\infty x^{\frac{1}{2} s - 1} \omega(x) dx 
	&= \\
		\frac{1}{s - 1} - \frac{1}{s} + \int _0 ^1 x^{\frac{s}{2} - \frac{3}{2}} \omega\cbr{\frac{1}{x}} dx + \int _1 ^\infty x^{\frac{s}{2} - 1} \omega(x) dx 
	&= \\
		\frac{1}{s - 1} - \frac{1}{s} + \int _1 ^\infty \cbr{x^{-\frac{s}{2} - \frac{1}{2}} + x^{\frac{s}{2} - 1}} \omega(x) dx &.
\end{aligned}
\end{equation*}
	Since the function $\omega$ has exponential decay at infinity the last integral defines an entire function. Thus the formula holds by analytic continuation for all $s \in \field{C}$ with simple poles at $s = 0$ and $s = 1$. The right-hand side remains unchanged when $s$ is replaced by $1 - s$. Hence we finally get
\begin{equation*}
	\xi(s) = -\frac{s(1 - s)}{2} \xi^*(s) = -\frac{(1 - s)s}{2} \xi^*(1 - s) = \xi(1 - s).
\end{equation*}
\end{proof}


\begin{lemma}\label{lem:XiMellin}
	Let $f(x) = 2 \theta(x^2)$. For $\sigma \in \cbr{0, 1}$ we have
\begin{equation*}
	\xi^*(s) = \int _0 ^\infty x^{s - 1} f(x) dx,
\end{equation*}
	hence $\xi^* = \fbr{\mathcal{M} f}$ and therefore by Mellin's inversion formula $f = \fbr{\mathcal{M}^{-1} \xi^*}$ whenever $\sigma \in \cbr{0, 1}$.
\end{lemma}
\begin{proof}
	This formula is essentially the same as stated in \cite{Edwards1974} [p. 213] after the substitution $u = \frac{1}{x}$.
\end{proof}


\begin{definition}
	Let
\begin{equation*}
\begin{aligned}
	&\Xi \colon \field{R} \to \field{C}, \\
	&\Xi(t) = \xi\cbr{\frac{1}{2} + it} = -\frac{1}{2}\cbr{t^2 + \frac{1}{4}} \pi ^{-\frac{1}{4} - \frac{1}{2} it} \Gamma\cbr{\frac{1}{4} + \frac{1}{2} it} \zeta\cbr{\frac{1}{2} + it}
\end{aligned}
\end{equation*}
	be the restriction of the xi function to $\sigma = \frac{1}{2}$.
\end{definition}


\begin{lemma}
	$\Xi$ is even.
\end{lemma}
\begin{proof}
	Let $t \in \field{R}$. We have
\begin{equation*}
	\Xi(-t) = \xi\cbr{\frac{1}{2} - it} = \xi\cbr{1 - \cbr{\frac{1}{2} + it}} = \xi\cbr{\frac{1}{2} + it} = \Xi(t).
\end{equation*}
\end{proof}


\begin{lemma}
	$\Xi$ is a real function.
\end{lemma}
\begin{proof}
	We first recall Corollary~\ref{cor:ThetaIdentity} which immediately yields the symmetric form
\begin{equation*}
	x^{-\frac{1}{4}} \theta\cbr{\frac{1}{x}} = x^{\frac{1}{4}} \theta(x).
\end{equation*}
	Plugging this into the integral form of the functional equation of $\xi^*$ from Lemma~\ref{lem:XiMellin} for $\sigma \in \cbr{0, 1}$ gives
\begin{equation*}
	\xi^*(s) = \int _0 ^\infty x^{\frac{1}{2}\cbr{s - \frac{1}{2}}} x^{\frac{1}{4} - 1} \theta(x) dx = \int _0 ^\infty x^{-\frac{1}{2}\cbr{s - \frac{1}{2}}} x^{\frac{1}{4} - 1} \theta(x) dx.
\end{equation*}
	By addition of the two sides we obtain
\begin{equation*}
	\xi^*(s) = \int _0 ^\infty \cosh\cbr{\frac{1}{2} \cbr{s - \frac{1}{2}} \log(x)} x^{\frac{1}{4} - 1} \theta(x) dx.
\end{equation*}
	Further, by applying the addition theorem
\begin{equation*}
	\cosh(v + w) = \cosh(v)\cosh(w) + \sinh(v) \sinh(w)
\end{equation*}	
	for all $v,w \in \field{C}$ we may deduce that
\begin{equation*}
\begin{aligned}	
	\xi^*(s) 
		&= \int _0 ^\infty \cosh\cbr{\frac{1}{2} \cbr{\sigma - \frac{1}{2}} \log(x)} \cos\cbr{\frac{t}{2} \log(x)} x^{\frac{1}{4} - 1} \theta(x) dx \\
		&+ i\int _0 ^\infty \sinh\cbr{\frac{1}{2} \cbr{\sigma - \frac{1}{2}} \log(x)} \sin\cbr{\frac{t}{2} \log(x)} x^{\frac{1}{4} - 1} \theta(x) dx.
\end{aligned}
\end{equation*}
	We now in fact observe that
\begin{equation*}
\begin{aligned}	
	\Im(\Xi(t)) 
	&= \lim _{\sigma \to \frac{1}{2}} \Im(\xi(\sigma + it)) \\
	&= \lim _{\sigma \to \frac{1}{2}} \Im\cbr{\frac{(\sigma + it)(\sigma + it - 1)}{2}\xi^*(\sigma + it)} = 0,
\end{aligned}
\end{equation*}
	hence $\Xi$ is entirely real.
\end{proof}


\section{Roots of $\zeta$ on the Critical Line}


\begin{proposition}
	We have the identity
\begin{equation*}	
	\int _0 ^{\infty} \frac{\Xi(t)}{t^2 + \frac{1}{4}} \cos(xt) dt = \frac{\pi}{2} (e^{\frac{x}{2}} - 2e^{-\frac{x}{2}} \omega(e^{-2x})).
\end{equation*}
\end{proposition}
\begin{proof}
	Let $f(t) = \lambda(it) \lambda(-it)$ such that $\abs{\lambda(it)}^2 = \lambda(it) \lambda(-it)$ where $\lambda$ is analytic and let
\begin{equation*}
	\Phi(x) = \int _0 ^{\infty} f(t) \Xi(t) \cos(xt) dt.	
\end{equation*}
	Since $\Xi$ is even and by substituting $y = e^x$ we may write
\begin{equation*}
\begin{aligned}
	\Phi(x) 
		&= \int _0 ^\infty \lambda(it) \lambda(-it) \Xi(t) \frac{1}{2}(y^{it} + y^{-it}) dt \\
		&= \frac{1}{2} \int _{-\infty} ^\infty \lambda(it) \lambda(-it) \Xi(t) y^{it} dt \\
		&= \frac{1}{2} \int _{-\infty} ^\infty \lambda(it) \lambda(-it) \xi\cbr{\frac{1}{2} + it} y^{it} dt
\end{aligned}
\end{equation*}
	Again the substitution $s = \frac{1}{2} + it$ gives
\begin{equation*}
\begin{aligned}
	\Phi(x) 
		&= \frac{1}{2 i\sqrt{y}} \int _{\frac{1}{2} -i\infty} ^{\frac{1}{2} + i\infty} \lambda\cbr{s - \frac{1}{2}} \lambda\cbr{\frac{1}{2} - s} \xi(s) y^{s} dt \\
		&= \frac{1}{2 i\sqrt{y}} \int _{\frac{1}{2} -i\infty} ^{\frac{1}{2} + i\infty} \lambda\cbr{s - \frac{1}{2}} \lambda\cbr{\frac{1}{2} - s} s(s - 1) \Gamma\cbr{\frac{s}{2}} \pi ^{-\frac{s}{2}} \zeta(s) y^{s} ds.
\end{aligned}
\end{equation*}
	Finally setting $\lambda(s) = \frac{1}{s + \frac{1}{2}}$ together with Lemma~\ref{lem:XiMellin} yields 
\begin{equation*}
\begin{aligned}
	\Phi(x)
		&= -\frac{1}{4 i\sqrt{y}} \int _{\frac{1}{2} - i\infty} ^{\frac{1}{2} + i\infty} \Gamma\cbr{\frac{s}{2}} \pi ^{-\frac{s}{2}} \zeta(s) y^{s} ds \\
		&= -\frac{\pi}{2 \sqrt{y}} \cbr{\frac{1}{2 i \pi} \int _{\frac{1}{2} - i\infty} ^{\frac{1}{2} + i\infty} \xi^*(s) \cbr{\frac{1}{y}}^{-s} ds} \\
		&= -\frac{\pi}{\sqrt{y}} \theta\cbr{\frac{1}{y^2}} \\
		&= -\frac{\pi}{\sqrt{y}} \omega\cbr{\frac{1}{y^2}} + \frac{1}{2} \pi \sqrt{y}.
\end{aligned}
\end{equation*}
	After re-substituting we receive the identity we wanted to show.
\end{proof}


\begin{theorem}
	$\zeta$ has infinitely many roots on the critical line $\sigma = \frac{1}{2}$.
\end{theorem}
\begin{proof}
	We substitute $x = -i \alpha$ and plug it into the previously shown identity
\begin{equation*}
\begin{aligned}
	\frac{2}{\pi} \int _0 ^{\infty} \frac{\Xi(t)}{t^2 + \frac{1}{4}} \cosh(\alpha t) dt 
	&= e^{-\frac{i \alpha}{2}} - 2e^{\frac{i \alpha}{2}} \omega(e^{2 i \alpha}) \\
	&= 2 \cos\cbr{\frac{\alpha}{2}} - 2e^{\frac{i \alpha}{2}} \cbr{\frac{1}{2} + \omega(e^{2 i \alpha})}.
\end{aligned}
\end{equation*}
	By Corollary~\ref{pro:ZetaEst} we know that $\zeta\cbr{\frac{1}{2} + it} = \mathcal{O}(t^A)$ as well as $\Xi(t) = \mathcal{O}(t^A e^{-\frac{1}{4} \pi t})$ for some constant $A > 0$ whenever $\abs{t} \geq 1$. This and the fact that $\cosh(\alpha t) = \mathcal{O}(e^{\alpha t})$ allows us to interchange integral and differentiation for $\alpha < \frac{1}{4} \pi$ according to the dominated convergence theorem. We then differentiate both sides $2n$-times with respect to $\alpha$ and receive
\begin{equation*}
\begin{aligned}	
	\frac{2}{\pi} \int _0 ^{\infty} \frac{\Xi(t)}{t^2 + \frac{1}{4}} t^{2n}
		&\cosh(\alpha t) dt \\
		&= \frac{(-1)^{n} \cos(\frac{\alpha}{2})}{2^{2n - 1}} - 2 \cbr{\frac{d}{d\alpha}}^{2n} \bbr{e^{\frac{i \alpha}{2}} \cbr{\frac{1}{2} + \omega(e^{2 i \alpha})}}.
\end{aligned}
\end{equation*}
	In order to show that the last term vanishes as $\alpha \to \frac{\pi}{4}$ for every fixed $n$ we use Corollary~\ref{cor:ThetaIdentity} and may write
\begin{equation*}
\begin{aligned}
	\omega(i + \delta) = \sum _{n=1} ^{\infty} e^{-n^2 \pi (i + \delta)}
		&= \sum _{n=1} ^{\infty} (-1)^n e^{-n^2 \pi \delta} \\
		&= 2 \omega(4 \delta) - \omega(\delta) \\
		&= \frac{1}{\sqrt \delta} \omega\cbr{\frac{1}{4\delta}} - \frac{1}{\sqrt \delta} \omega\cbr{\frac{1}{\delta}} - \frac{1}{2}.
\end{aligned}
\end{equation*}
	Thus it follows easily that $\omega(x) + \frac{1}{2}$ and all its derivatives tend to $0$ as $x \to i$ along every route in an angle $\abs{\arg(x - i)} < \frac{\pi}{2}$. Therefore we have shown that
\begin{equation*}
	\lim _{\alpha \to \frac{1}{4} \pi} \frac{2}{\pi} \int _0 ^{\infty} \frac{\Xi(t)}{t^2 + \frac{1}{4}} t^{2n} \cosh(\alpha t) dt = \frac{(-1)^{n} \cos(\frac{\pi}{8})}{2^{2n - 1}}.
\end{equation*}
	Let us assume that $\Xi$ were ultimately of one sign, say positive for $t \geq T$ where $T \in \field{R}$. Then we write
\begin{equation*}
	\lim _{\alpha \to \frac{1}{4} \pi} \frac{2}{\pi} \int _T ^{\infty} \frac{\Xi(t)}{t^2 + \frac{1}{4}} t^{2n} \cosh(\alpha t) dt \eqqcolon L.
\end{equation*}
	Hence for all $T' > T$ and $\alpha < \frac{\pi}{4}$ we have
\begin{equation*}
	\frac{2}{\pi} \int _T ^{T'} \frac{\Xi(t)}{t^2 + \frac{1}{4}} t^{2n} \cosh(\alpha t) dt \leq L,
\end{equation*}
	Thus for $\alpha = \frac{\pi}{4}$ we get
\begin{equation*}
	\frac{2}{\pi} \int _T ^{T'} \frac{\Xi(t)}{t^2 + \frac{1}{4}} t^{2n} \cosh\cbr{\frac{\pi}{4} t} dt \leq L.
\end{equation*}
	Therefore the integral
\begin{equation*}
	\frac{2}{\pi} \int _0 ^{\infty} \frac{\Xi(t)}{t^2 + \frac{1}{4}} t^{2n} \cosh\cbr{\frac{\pi}{4} t} dt
\end{equation*}
	is convergent and therefore the integral is uniformly convergent with respect to $0 \leq \alpha \leq \frac{\pi}{4}$ and it follows that
\begin{equation*}
	\int _0 ^{\infty} \frac{\Xi(t)}{t^2 + \frac{1}{4}} t^{2n} \cosh\cbr{\frac{\pi}{4} t} dt = \frac{(-1)^{n} \pi \cos\cbr{\frac{\pi}{8}}}{2^{2n}}.
\end{equation*}
	Now we choose $n$ to be odd. The right-hand side is negative and therefore
\begin{equation*}
\begin{aligned}
	\int _T ^{\infty} \frac{\Xi(t)}{t^2 + \frac{1}{4}} t^{2n} \cosh\cbr{\frac{\pi}{4} t} dt 
		&< - \int _0 ^{T} \frac{\Xi(t)}{t^2 + \frac{1}{4}} t^{2n} \cosh\cbr{\frac{\pi}{4} t} dt \\
		&< KT^{2n},
\end{aligned}
\end{equation*}
	where $K$ is independent of $n$. But by hypothesis there is a positive $m = m(T)$ such that
\begin{equation*}
	\frac{\Xi(t)}{t^2 + \frac{1}{4}} \geq m
\end{equation*}
	for $2T \leq t \leq 2T + 1$. Hence
\begin{equation*}
	\int _T ^{\infty} \frac{\Xi(t)}{t^2 + \frac{1}{4}} t^{2n} \cosh(\frac{1}{4} \pi t) dt \geq \int _{2T} ^{2T + 1} mt^{2n} dt \geq m(2T)^{2n}.
\end{equation*}
	Thus
\begin{equation*}
	m2^{2n} < K,
\end{equation*}
	which apparently is false for sufficiently large $n$. That proves our theorem by contradiction.
\end{proof}