\documentclass[11pt]{report}
\usepackage[utf8]{inputenc}
\usepackage{amssymb,amsthm,mathtools}
\usepackage{hyperref,graphics}
\usepackage{bigints}
\usepackage{todonotes}
\usepackage{natbib}

\usetikzlibrary{decorations.markings}
\graphicspath{{images/}}

\newcommand{\eps}{\varepsilon}
\newcommand{\hsp}{\hspace{0.3cm}}

\delimitershortfall-1sp
\newcommand{\field}[1]{\mathbb{#1}}

\newcommand{\abs}[1]{\left|#1\right|}
\newcommand{\norm}[1]{\left\lVert#1\right\rVert}

\newcommand{\cbr}[1]{\left(#1\right)}
\newcommand{\bbr}[1]{\left[#1\right]}
\newcommand{\cbbr}[1]{\left(#1\right]}
\newcommand{\bcbr}[1]{\left[#1\right)}
\newcommand{\fbr}[1]{\left\{#1\right\}}

\def\res{\mathop{Res}}

\newcommand{\mysetminusD}{\hbox{\tikz{\draw[line width=0.6pt,line cap=round] (3pt,0) -- (0,6pt);}}}
\newcommand{\mysetminusT}{\mysetminusD}
\newcommand{\mysetminusS}{\hbox{\tikz{\draw[line width=0.45pt,line cap=round] (2pt,0) -- (0,4pt);}}}
\newcommand{\mysetminusSS}{\hbox{\tikz{\draw[line width=0.4pt,line cap=round] (1.5pt,0) -- (0,3pt);}}}
\newcommand{\mysetminus}{\mathbin{\mathchoice{\mysetminusD}{\mysetminusT}{\mysetminusS}{\mysetminusSS}}}

\DeclareMathOperator{\dist}{\text{dist}}

\newtheorem{definition}{Definition}[chapter]
\newtheorem{theorem}{Theorem}[section]
\newtheorem{proposition}{Proposition}[section]
\newtheorem{corollary}{Corollary}[theorem]
\newtheorem{lemma}[theorem]{Lemma}
\newtheorem*{remark}{Remark}

\begin{document}

\title{
	{Riemann Zeta Function - The Critical Strip}\\
	{\large University of Vienna}\\
	{\vspace{3ex}\includegraphics[width=15ex]{University.png}}
}
\author{Florian Jesacher}
\date{September 21, 2016}
\maketitle

\begin{abstract}
By taking the complex logarithm of the Euler product representation of the Riemann zeta function we first show that said function doesn't vanish on the lines with the real part being zero and one. Furthermore we manage to establish growth estimations for the zeta function, its reciprocal function and its derivative under certain conditions. After introducing the prime-counting function pi and Tchebychev's psi function we then deduce their asymptotics by defining two auxiliary functions in order to complete our proof of the prime number theorem. Moreover, once we defined the Mellin transformation and stated the Mellin inversion theorem we use the xi function to derive a new formula which finally provides us the necessary properties for the conclusive evidence that the Riemann zeta function has infinitely many roots on the critical line.
\end{abstract}

\tableofcontents

\chapter{Introduction}
The Riemann zeta function has many interesting properties in the critical strip that have a huge impact on the analytic number theory, especially concerning the distribution of prime numbers. We want to have a closer look at its zero-free regions and find good estimates of the function's growth under certain circumstances. These will be needed later on in order to prove the prime number theorem and to show that the Riemann zeta function has infinitely many zeros on the critical line. \\
This essay is a continuation of \cite{Jesacher2016} and is mainly based on \cite{Stein2003} and \cite{Titchmarsh1986}. \\
Henceforth let $s \in \field{C}$ be a complex number where $\sigma \coloneqq \Re(s)$ and $t \coloneqq \Im(s)$ and if nothing else is stated let $p \in \field{N}$ be a prime number. The space of all holomorphic functions on a domain $S \subseteq \field{C}$ shall be denoted by $\mathcal{H}(S) \coloneqq \fbr{f \in \field{C} ^S \colon f \textit{ holomorphic}}$.


\section{Properties of $\zeta$ in the critical strip}
First of all we want to draw some conclusions about the function's behaviour at the borders of the critical strip and show the absence of zeros on the lines $\sigma = 0$ and $\sigma = 1$.


\begin{lemma}\label{lem:LogEulerProd}
	If $\sigma > 1$ then
\begin{equation*}
	\log(\zeta(s)) = \sum _{p,m} \frac{p^{-ms}}{m}.
\end{equation*}
\end{lemma}
\begin{proof}
	Let $x \in \cbr{0, 1}$. Using the power series expansion for the logarithm
\begin{equation*}
	\log\cbr{\frac{1}{1 - x}} = -\log(1 + (-x)) = \sum _{m = 1} ^\infty \cbr{\frac{x^m}{m}}.
\end{equation*}
	and applying this to $\zeta$'s absolutely convergent Euler product representation while assuming that $s \in \cbr{0 , \infty}$ yields
\begin{equation*}
\begin{aligned}	
	\log(\zeta(s)) 
		&= \log\cbr{\prod _p \frac{1}{1 - p^{-s}}} \\
		&= \sum _p \log\cbr{\frac{1}{1 - p^{-s}}}
		= \sum _{p,m} \frac{p^{-ms}}{m} = \sum _{n=1} ^{\infty} c_n n^{-s},
\end{aligned}
\end{equation*}
	where
\begin{equation*}
    c_n =
    \left\{
    	\begin{array}{ll}
        	\frac{1}{m}, & \text{for } n = p^m, \\
        	0, & \text{otherwise.}
        \end{array}
	\right.
\end{equation*}
	By analytic continuation this also holds for $s \in \field{C}$ with $\sigma > 1$. In particular if $U = \fbr{s \in \field{C} \colon \sigma > 1}$ then $\zeta \in \mathcal{H}(U)$ where $U$ is simply connected and $\zeta(s) \neq 0$ for all $s \in U$. Hence there exists a $g \in \mathcal{H}(U)$ such that $\zeta(s) = e^{g(s)}$. Therefore $\log(\zeta(s))$ is well defined for all $s \in U$.
\end{proof}


\begin{lemma}\label{lem:CosAddTh}
	If $\theta \in \field{R}$ then
\begin{equation*}
	3 + 4 \cos(\theta) + \cos(2 \theta) \geq 0.
\end{equation*}
\end{lemma}
\begin{proof}
	The application of the well known trigonometric identity
\begin{equation*}
	\cos(\theta)^2 = \frac{1 + \cos(2 \theta)}{2}
\end{equation*}
	immediately leads to
\begin{equation*}
	3 + 4 \cos(\theta) + \cos(2 \theta) = 2 + 4 \cos(\theta) + 2 \cos(\theta)^2 = 2(1 + \cos(\theta))^2 \geq 0.
\end{equation*}
\end{proof}


\begin{corollary}\label{cor:LogZeta}
	If $\sigma > 1$ then
\begin{equation*}
	\log\cbr{\abs{\zeta^3(\sigma) \zeta^4(\sigma + it) \zeta(\sigma + 2it)}} \geq 0.
\end{equation*}
\end{corollary}
\begin{proof}
	We write
\begin{equation*}
\begin{aligned}	
	\Re(n^{-s}) 
		&= \Re(e^{-(\sigma + it) \log(n)}) \\
		&= e^{-\sigma \log(n)} \cos(t \log(n)) 
		= n^{-\sigma} \cos(t \log(n))
\end{aligned}
\end{equation*}
	and for $r > 0$ we have
\begin{equation*}
	\Re(\log(r e^{i \theta})) = \log(r) = \log\cbr{\abs{r e^{i\theta}}}.
\end{equation*}
	Therefore Lemma~\ref{lem:LogEulerProd} and Lemma~\ref{lem:CosAddTh} directly give us our final result
\begin{equation*}
\begin{aligned}
	\log\cbr{\abs{\zeta^3(\sigma) \zeta^4(\sigma + it) \zeta(\sigma + 2it)}} &= \\
	3\log(\abs{\zeta(\sigma)}) + 4\log(\abs{\zeta(\sigma + it)}) + \log(\abs{\zeta(\sigma + 2it)}) &= \\
	3\Re(\log(\zeta(\sigma))) + 4\Re(\log(\zeta(\sigma + it))) + \Re(\log(\zeta(\sigma + 2it))) &= \\
	\sum _{n=1} ^{\infty} c_n n^{-\sigma}(3 + 4\cos(t \log(n)) + \cos(2 t \log(n))) &\geq 0.
\end{aligned}
\end{equation*}
\end{proof}


\begin{theorem}
	It holds that $\zeta(1 + it) \neq 0$ for all $t \in \field{R}$.
\end{theorem}
\begin{proof}
	Let us suppose there exists a constant $t_0 \neq 0$ such that $\zeta(1 + t_0) = 0$. We know that $\zeta$ is holomorphic on $\field{C} \mysetminus \{1\}$, particularly in $s = 1 + it_0$. Hence
\begin{equation*}
	\lim\limits_{h \to 0} \frac{\zeta(1 + it_0 + h) - \zeta(1 + it_0)}{h} = \lim\limits_{\sigma \to 1} \frac{\zeta(\sigma + it_0)}{\sigma - 1} \in \field{C}.
\end{equation*}
	Thus
\begin{equation*}
	\abs{\zeta^4(\sigma + it_0)} = \mathcal{O}\cbr{\abs{\sigma - 1}^4} \text{ as } \sigma \to 1.
\end{equation*}
	Since $\zeta$ has a simple pole in $s = 1$ we get
\begin{equation*}
	\lim\limits_{\sigma \to 1}\zeta(\sigma) (\sigma - 1) \in \field{C}.
\end{equation*}
	Thus we find that
\begin{equation*}
	 \abs{\zeta^3(\sigma)} = \mathcal{O}\cbr{\abs{\sigma - 1}^{-3}} \text{ as } \sigma \to 1.
\end{equation*}
	Since $\zeta$ is also holomorphic in $s = 1 + 2it_0$ the function is locally bounded there, therefore we have
\begin{equation*}
	 \abs{\zeta(\sigma + 2it_0)} = \mathcal{O}\cbr{1} \text{ as } \sigma \to 1.
\end{equation*}
	Hence we get
\begin{equation*}
	 \lim\limits_{\sigma \to 1} \abs{\zeta(\sigma)^3 \zeta(\sigma + it_0)^4 \zeta(\sigma + 2it_0)} = 0.
\end{equation*}
	But this contradicts Corollary~\ref{cor:LogZeta} since $\log(x) < 0$ for all $x \in \cbr{0 , 1}$.
\end{proof}


\begin{corollary}
	It holds that $\zeta(it) \neq 0$ for all $t \in \field{R}$.
\end{corollary}
\begin{proof}
	This proof relies on the fact that the $\xi$ function (see Definition~\ref{def:XiDefinition}) satisfies the symmetry $\xi(s) = \xi(1 - s)$ (see Proposition~\ref{pro:XiIdentity}). This, and the fact that only $\zeta$ can produce zeros in $\xi$, immediately yields
\begin{equation*}
	\zeta(1 + it) \neq 0 \implies \xi(1 + it) = \xi(-it) \neq 0 \implies \zeta(-it) \neq 0,	
\end{equation*}
	for all $t \in \field{R}$.
\end{proof}


\section{Estimates involving $\zeta$}
We want to study the growth of $\zeta$, $\zeta'$ and $\zeta^{-1}$ in the half-plane $\sigma > 0$ in more detail in order to find good estimates for later applications.

\begin{proposition}
	There exists a sequence of entire functions $\fbr{\delta_n(s)} _{n = 1} ^\infty$ that satisfy the estimate $\abs{\delta_n(s)} \leq \frac{\abs{s}}{n^{\sigma + 1}}$ and such that
\begin{equation}\label{equ:SumDelta}
	\sum _{1 \leq n < N} \frac{1}{n^s} - \int _1 ^N \frac{dx}{x^s} = \sum _{1 \leq n < N} \delta_n(s),
\end{equation}
	whenever $N$ is an integer $> 1$ and $\sigma > 0$.
\end{proposition}
\begin{proof}
	In order to prove our claim we compare $\sum _{1 \leq n < N} n^{-s}$ with $\sum _{1 \leq n < N} \int _n ^{n + 1} x^{-s} dx$, and set
\begin{equation}\label{equ:DeltaN}
	\delta_n(s) = \int _n ^{n + 1} \cbr{\frac{1}{n^s} - \frac{1}{x^s}} dx.
\end{equation}
	Let $f(x) = x^{-s}$. We have
\begin{equation*}
\begin{aligned}	
	\abs{\frac{1}{n^s} - \frac{1}{x^s}} = \abs{\int _n ^x f'(y) dy} 
	&\leq \int _n ^x \abs{f'(y)} dy \\
	&\leq \sup _{y \in \bbr{n, x}} \abs{f'(y)} = \abs{f'(n)} = \frac{\abs{s}}{n^{\sigma + 1}},
\end{aligned}
\end{equation*}
	whenever $x \in \bbr{n, n + 1}$. Therefore $\abs{\delta_n(s)} \leq \frac{\abs{s}}{n^{\sigma + 1}}$, and since
\begin{equation*}
	\int _1 ^N \frac{dx}{x^s} = \sum _{1 \leq n < N} \int _n ^{n + 1} \frac{dx}{x^s},
\end{equation*}
	our proof is complete.
\end{proof}


\begin{corollary}\label{cor:ZetaCont}
	For $\sigma > 0$ we have
\begin{equation*}
	\zeta(s) - \frac{1}{s - 1} = H(s),
\end{equation*}
	where $H(s) = \sum _{n = 1} ^\infty \delta_n(s)$ is holomorphic in the half-plane $\sigma > 0$. 
\end{corollary}
\begin{proof}
	We first assume that $\sigma > 1$. When $N \to \infty$ in formula~(\ref{equ:SumDelta}) we observe that by the estimate $\abs{\delta_n(s)} \leq \frac{\abs{s}}{n^{\sigma + 1}}$ we have uniform convergence of the series $\sum _{n = 1} ^\infty \delta_n(s)$ on compact sets in any half plane $\sigma \geq \delta$ when $\delta > 0$. Since $\sigma > 1$ the series $\sum _{n = 1} ^{\infty} n^{-s}$ converges to $\zeta(s)$. This proves our assertion when $\sigma > 1$. The uniform convergence shows that $\sum _{n = 1} ^\infty \delta_n(s)$ is holomorphic when $\sigma > 0$ and thus shows that $\zeta$ is extendable to that half-plane and that the identity continues to hold there.
\end{proof}


\begin{proposition}\label{pro:ZetaEst}
	For each $\sigma_0 \in \bbr{0 , 1}$ and every $\eps > 0$, there exists a constant $c_\eps$, such that
\begin{equation}\label{equ:ZetaEst1}
	\abs{\zeta(s)} \leq c_\eps \abs{t}^{1 - \sigma_0 + \eps}, \textit{ if } \sigma_0 \leq \sigma \textit{ and } \abs{t} \geq 1
\end{equation}
	and
\begin{equation}\label{equ:ZetaEst2}
	\abs{\zeta'(s)} \leq c_\eps \abs{t}^\eps, \textit{ if } 1 \leq \sigma \textit{ and } \abs{t} \geq 1.
\end{equation}
\end{proposition}
\begin{proof}
	For the proof we recall the estimate $\abs{\delta_n(s)} \leq \frac{\abs{s}}{n^{\sigma + 1}}$. We also have the estimate $\abs{\delta_n(s)} \leq \frac{2}{n^\sigma}$, which follows from the expression for $\delta_n$ given by~(\ref{equ:DeltaN}) and the fact that $\abs{n^{-s}} = n^{-\sigma}$ and $\abs{x^{-s}} \leq n^{-\sigma}$ if $x \geq n$. We then combine these two estimates for $\abs{\delta_n(s)}$ via the observation that $A = A^\delta A^{1 - \delta}$, to obtain the bound
\begin{equation*}
	\abs{\delta_n(s)} \leq \cbr{\frac{\abs{s}}{n^{\sigma_0 + 1}}}^\delta \cbr{\frac{2}{n^{\sigma_0}}}^{1 - \delta} \leq \frac{2 \abs{s}^\delta}{n^{\sigma_0 + \delta}},
\end{equation*}
	as long as $\delta \geq 0$. Now choose $\delta = 1 - \sigma_0 + \eps$ and apply the estimate to Corollary~\ref{cor:ZetaCont}. Then, with $\sigma \geq \sigma_0$, we find
\begin{equation*}
	\abs{\zeta(s)} \leq \abs{\frac{1}{s - 1}} + 2 \abs{s}^{1 - \sigma_0 + \eps} \sum _{n = 1} ^\infty \frac{1}{n^{1 + \eps}},
\end{equation*}
	and conclusion~(\ref{equ:ZetaEst1}) is proved. By the Cauchy integral formula,
\begin{equation*}
	\zeta'(s) = \frac{1}{2 \pi r} \int _0 ^{2 \pi} \zeta(s + re^{i \theta}) e^{-i \theta} d\theta,
\end{equation*}
	where the integration is taken over a circle of radius $r$ centred at $s$. Now choose $r = \eps$ and observe that this circle lies in the half-plane $\sigma \geq 1 - \eps$, and so the estimate~(\ref{equ:ZetaEst2}) follows as a consequence of the estimate~(\ref{equ:ZetaEst1}) on replacing $2 \eps$ by $\eps$.
\end{proof}


\begin{proposition}
	For every $\eps > 0$ there exists a constant $c_\eps$, such that $\frac{1}{\abs{\zeta(s)}} \leq c_\eps \abs{t}^\eps$, if $\sigma \geq 1$ and $\abs{t} \geq 1$.
\end{proposition}
\begin{proof}
	By Corollary~\ref{cor:LogZeta} we have that
\begin{equation*}
	\abs{\zeta^3(\sigma) \zeta^4(\sigma + it) \zeta(\sigma + 2it)} \geq 1,
\end{equation*}
	whenever $\sigma > 1$. Using our previous estimate~(\ref{equ:ZetaEst1}) for $\zeta$ we find that
\begin{equation*}
	\abs{\zeta^4(\sigma + it)} \geq c \abs{\zeta^{-3}(\sigma)} \abs{t}^{-\eps} \geq c' (\sigma - 1)^3 \abs{t}^{-\eps}
\end{equation*}
	for all $\sigma \geq 1$ and $\abs{t} \geq 1$. Thus
\begin{equation*}
	\abs{\zeta(\sigma + it)} \geq c' (\sigma - 1)^{\frac{3}{4}} \abs{t}^{-\frac{\eps}{4}}.
\end{equation*}
	We now consider two cases, depending on whether $\sigma - 1 \geq A\abs{t}^{-5 \eps}$ holds for an appropriate constant $A \in \field{R}_{> 0}$ or not. If this inequality does hold it follows
\begin{equation*}
	\abs{\zeta(\sigma + it)} \geq A' \abs{t}^{-4 \eps}.
\end{equation*}
	By replacing $4 \eps$ by $\eps$ we can conclude the proof in this case. If, however, $\sigma - 1 < A\abs{t}^{-5 \eps}$ then we first choose $\sigma' > \sigma$ with $\sigma' - 1 = A \abs{t}^{-5 \eps}$. The triangle inequality then implies
\begin{equation*}
	\abs{\zeta(\sigma + it)} \geq \abs{\zeta(\sigma' + it)} - \abs{\zeta(\sigma' + it) - \zeta(\sigma + it)},
\end{equation*}
	and combined with the previous estimate~(\ref{equ:ZetaEst2}) for $\zeta'$ we obtain
\begin{equation*}
\begin{aligned}	
	\abs{\zeta(\sigma' + it) - \zeta(\sigma + it)} = \abs{\int _\sigma ^{\sigma'} \zeta'(x + it) dx}
	&\leq \int _\sigma ^{\sigma'} \abs{\zeta'(x + it)} dx \\ 
	&\leq c'' (\sigma' - \sigma) \abs{t}^{\eps} \\
	&\leq c''(\sigma' - 1)\abs{t}^{\eps}.
\end{aligned}
\end{equation*}
	These observations, together with our result above where we set $\sigma = \sigma'$, show that
\begin{equation*}
	\abs{\zeta(\sigma + it)} \geq c'(\sigma' - 1)^{\frac{3}{4}}\abs{t}^{-\frac{\eps}{4}} - c''(\sigma' - 1)\abs{t}^{\eps}.
\end{equation*}
	We can now choose $A = \cbr{\frac{c'}{2 c''}}^4$, and recall that $\sigma' - 1 = A \abs{t}^{-5 \eps}$. This gives precisely
\begin{equation*}
	c'(\sigma' - 1)^{\frac{3}{4}} \abs{t}^{-\frac{\eps}{4}} = 2 c''(\sigma' - 1) \abs{t}^{\eps}
\end{equation*}
	and therefore
\begin{equation*}
	\abs{\zeta(\sigma + it)} \geq A''\abs{t}^{-4 \eps}.
\end{equation*}
	By replacing $4 \eps$ by $\eps$ the desired inequality is established and therefore the proof is complete.
\end{proof}


\begin{corollary}\label{cor:ZetaQuotEst}
	For every $\eps > 0$ we have $\abs{\frac{\zeta'(s)}{\zeta(s)}} \leq c_\eps \abs{t}^\eps$ for $c_\eps \in \field{R}_{> 0}$ if $\sigma \geq 1$ and $\abs{t} \geq 1$.
\end{corollary}
\begin{proof}
	Our desired result immediately follows from combining the previous estimates for $\zeta'$ and $\frac{1}{\zeta}$.
\end{proof}


\section{Mellin Transform}


\begin{definition}
	The function defined by
\begin{equation*}
	\fbr{\mathcal{M}f}(s) = \int _0 ^\infty x^{s - 1} f(x) dx.
\end{equation*}
	is called the Mellin transform of the function $f$. Conversely
\begin{equation*}
	\fbr{\mathcal{M}^{-1} \varphi}(x) = \frac{1}{2 \pi i} \int _{c - i \infty} ^{c + i \infty} x^{-s} \varphi(s) ds.
\end{equation*}
	defines the inverse Mellin transform.
\end{definition}


\begin{theorem}[Mellin's Inversion Formula]
	Let $\varphi(s)$ be analytic for $\sigma \in \cbr{a, b}$ and $\varphi(s) \to 0$ uniformly as $t \to \pm\infty$ with its integral along the line $\sigma = c$ converging absolutely where $c \in \cbr{a, b}$. Then if
\begin{equation*}
	f(x) = \fbr{\mathcal{M}^{-1} \varphi}(x) = \frac{1}{2 \pi i} \int _{c - i \infty} ^{c + i\infty} x^{-s} \varphi(s) ds
\end{equation*}
	we have that
\begin{equation*}
	\varphi(s) = \fbr{\mathcal{M} f}(s) = \int _0 ^\infty x^{s - 1} f(x) dx.
\end{equation*}
	Conversely suppose $f$ is piecewise continuous on the positive real numbers and suppose the integral
\begin{equation*}
	\varphi(s) = \int _0 ^\infty x^{s - 1} f(x) dx
\end{equation*}
	is absolutely convergent for $\sigma \in \cbr{a, b}$. Then $f$ is recoverable via the inverse Mellin transform from its Mellin transform.
\end{theorem}
\begin{proof}
	This theorem follows from the inversion formula for the Laplace transform and the proof can, for example, be found in $\cite{McLachlan1953}$.
\end{proof}


\section{The Poisson Summation Formula}


\begin{definition}
	For each $a > 0$ we define $\mathcal{F}_a$ to be the space of all functions $f$ that satisfy the following two conditions:
\begin{enumerate}
	\item[(i)] The function $f$ is holomorphic on the horizontal strip
		\begin{equation*}
			S_a = \fbr{s \in \field{C} \colon \abs{t} < a}.
		\end{equation*}
	\item[(ii)] There exists a constant $A > 0$ such that
		\begin{equation*}
			\abs{f(s)} \leq \frac{A}{1 + \sigma^2} \hsp \textit{ for all } \sigma \in \field{R}, \abs{t} < a.
		\end{equation*}
\end{enumerate}
\end{definition}


\begin{theorem}[Poisson Summation Formula]\label{the:Poisson}
	For $f \in \mathcal{F}_a$ for some $a > 0$, the Poisson summation may be stated as
\begin{equation*}
	\sum _{n = -\infty} ^\infty f(n) = \sum _{k = -\infty} ^\infty \hat{f}(k),
\end{equation*}
	where $\hat{f}$ is the Fourier transform of $f \big\vert _\field{R}$.
\end{theorem}
\begin{proof}
	The entire proof can be found in \cite{Stein2003}.
\end{proof}

\chapter{The Prime Number Theorem}
The prime number theorem is a very fundamental statement about the asymptotic behaviour of the distribution of prime numbers. Based on the fact which we established in the previous chapter, that the Riemann zeta function doesn't have zeros on the line $\sigma = 1$ and by introducing Tchebychev's psi-function together with two auxiliary functions, we will have enough tools to manage to prove said theorem.


\section{Tchebychev's $\phi$ function}


\begin{definition}
	Let $\mathcal{P} = \fbr{p \in \field{N} \colon \; p \textit{ prime number}}$. We define the function
\begin{equation*}
\begin{aligned}
	&\pi \colon \field{R} \to \field{R}, \\
	&\pi(x) = \abs{\fbr{p \in \mathcal{P}: p \leq x}}
\end{aligned}
\end{equation*}
	as the prime number counting function.
\end{definition}


\begin{definition}
	The function
\begin{equation*}
\begin{aligned}
	&\psi \colon \field{R} \to \field{R}, \\
	&\psi(x) = \sum _{p^m \leq x} \log(p),
\end{aligned}
\end{equation*}
	for some prime number $p$ and positive integer $m \in \field{N}_{> 0}$ is often referred to as Tchebychev's psi-function.
\end{definition}


\begin{lemma}
	We have
\begin{equation*}
	\psi(x) = \sum _{p < x} \bbr{\frac{\log(x)}{\log(p)}} \log(p).
\end{equation*}
	where $\bbr{u}$ denotes the greatest integer $\leq u$.
\end{lemma}
\begin{proof}
	First we define
\begin{equation*}
	\Lambda(n) =
    	\left\{
    		\begin{array}{ll}
        		\log(p), & \text{if } n = p^m,\\
        		0, & \text{otherwise},
        	\end{array}
		\right.
\end{equation*}
	then it's clear that
\begin{equation*}
	\psi(x) = \sum _{1 \leq n \leq x} \Lambda(n).
\end{equation*}
	With this observation and the fact that if $p^m \leq x$ then $m \leq \frac{\log(x)}{\log(p)}$ our formula follows immediately.
\end{proof}


\section{Asymptotics of $\psi$ and $\psi_1$}


\begin{lemma}
	If $\psi(x) \sim x$ as $x \to \infty$, then $\pi(x) \sim \frac{x}{\log(x)}$ as $x \to \infty$.
\end{lemma}
\begin{proof}
	By definition we have to prove the inequalities
\begin{equation*}
	1 \leq \liminf _{x \to \infty} \pi(x) \frac{\log(x)}{x}
\end{equation*}
	and
\begin{equation*}
	\limsup _{x \to \infty} \pi(x) \frac{\log(x)}{x} \leq 1.
\end{equation*}
	At first we can make the crude estimate
\begin{equation*}
	\psi(x) = \sum _{p < x} \bbr{\frac{\log(x)}{\log(p)}} \log(p) \leq \sum _{p < x} \frac{\log(x)}{\log(p)} \log(p) = \pi(x) \log(x)
\end{equation*}
	and dividing by $x$ yields
\begin{equation*}
	\frac{\psi(x)}{x} \leq \frac{\pi(x) \log(x)}{x}.
\end{equation*}
	This asymptotic condition $\psi(x) \sim x$ implies the first inequality. For the second inequality we fix $0 < \alpha < 1$ and note that
\begin{equation*}
	\psi(x) \geq \sum _{p \leq x} \log(p) \geq \sum _{x^\alpha < p < x} \log(p) \geq \cbr{\pi(x) - \pi(x^\alpha)} \log(x^\alpha).
\end{equation*}
	and therefore
\begin{equation*}
	\psi(x) + \alpha \pi(x^\alpha) \log(x) \geq \alpha \pi(x) \log(x).
\end{equation*}
	Dividing by $x$, noting that $\pi(x^\alpha) \leq x^\alpha$ and $\psi(x) \sim x$, gives
\begin{equation*}
	1 \geq \alpha \limsup _{x \to \infty} \pi(x) \frac{\log(x)}{x}.
\end{equation*}
	Since $\alpha < 1$ is arbitrary the proof is complete.
\end{proof}


\begin{definition}
	The function
\begin{equation*}
\begin{aligned}
	&\psi_1 \colon \field{R} \to \field{R}, \\
	&\psi_1(x) = \sum _{p^m \leq x} \log(p),
\end{aligned}
\end{equation*}
	is closely related to the $\psi$-function and more convenient to deal with.
\end{definition}


\begin{lemma}
	If $\psi_1(x) \sim \frac{x^2}{2}$ as $x \to \infty$, then $\psi(x) \sim x$ as $x \to \infty$.
\end{lemma}
\begin{proof}
	It suffices to show that $\psi(x) \sim x$. This can easily be seen from the fact that if $\alpha < 1 < \beta$, then
\begin{equation*}
	\frac{1}{(1 - \alpha) x} \int _{\alpha x} \psi(u) du \leq \psi(x) \leq \frac{1}{(\beta - 1) x} \int _{x} \psi(\beta u) du.
\end{equation*}
	The proof of this double inequality relies simply on the fact that $\phi$ is increasing. Consequently we find that
\begin{equation*}
	\psi(x) \leq \frac{1}{(\beta - 1)x} \cbr{\psi_1(\beta x) - \psi_1(x)},
\end{equation*}
	and therefore
\begin{equation*}
	\frac{\psi(x)}{x} \leq \frac{1}{\beta - 1} \cbr{\frac{\psi_1(\beta x)}{(\beta x)^2} \beta^2 - \frac{\psi_1(x)}{x^2}}.
\end{equation*}
	In turn this implies
\begin{equation*}
	\limsup _{x \to \infty} \frac{\psi(x)}{x} \leq \frac{1}{\beta - 1} \cbr{\frac{1}{2} \beta^2 - \frac{1}{2}} = \frac{1}{2} (\beta + 1).
\end{equation*}
	Since this result holds for all $\beta > 1$ , we have proved that
\begin{equation*}
	\limsup _{x \to \infty} \frac{\psi(x)}{x} \leq 1. 
\end{equation*}
	For showing
\begin{equation*}
	\liminf _{x \to \infty} \frac{\psi(x)}{x} \geq 1
\end{equation*}
	we can use a similar argument with $\alpha < 1$.
\end{proof}


\begin{lemma}
	For $\sigma > 1$ we have
\begin{equation*}
	\frac{\zeta'(s)}{\zeta(s)} = \sum _{n = 1} ^\infty \frac{\Lambda(n)}{n^s}.
\end{equation*}
\end{lemma}
\begin{proof}
	We know that for $\sigma > 1$ we have
\begin{equation*}
	\log(\zeta(s)) = \sum _{m,p} \frac{p^{-ms}}{m}.
\end{equation*}
	Differentiating both expressions leads to
\begin{equation*}
	\frac{\zeta'(s)}{\zeta(s)} = \sum _{m,p} p^{-ms} \log(p) = \sum _{n = 1} ^\infty \frac{\Lambda(n)}{n^s}.
\end{equation*}
\end{proof}


\begin{lemma}
	If $c > 0$, then
\begin{equation*}
	\frac{1}{2 \pi i} = \int _{c - i \infty} ^{c + i \infty} \frac{a^s}{s(s + 1)} ds = 
		\left\{
    		\begin{array}{ll}
        		0, &\text{if } 0 \leq a \leq 1,\\
        		1 - \frac{1}{a}, &\text{if } 1 \leq a.
        	\end{array}
		\right.
\end{equation*}
\end{lemma}
\begin{proof}
\begin{figure}[!htb]
\begin{minipage}[c]{0.5\textwidth}
\centering
\begin{equation*}
\begin{aligned}
	&\begin{aligned}
		C(T) \colon \bbr{0, \pi} &\to \field{C}, \\ \phi &\mapsto c + Te^{i (\phi + \frac{\pi}{2}) },
	\end{aligned} \\
	&\begin{aligned}
		S(T) \colon \bbr{-T, T} &\to \field{C}, \\ t &\mapsto c + it
	\end{aligned}
\end{aligned}
\end{equation*}
\end{minipage}
\begin{minipage}[c]{0.5\textwidth}
\raggedleft
\begin{tikzpicture}
	[decoration={
		markings,
		mark=at position 0.7cm with {\arrow[line width=1pt]{>}},
		mark=at position 4.5cm with {\arrow[line width=1pt]{>}}
	}]
	% The axes
	\draw[help lines, very thin, ->] (-3,0) -- (3,0) coordinate (xaxis);
	\draw[help lines, very thin, ->] (0,-2) -- (0,2) coordinate (yaxis);
	\node[below, gray] at (xaxis) {$\Re(z)$};
	\node[left, gray] at (yaxis) {$\Im(z)$};

	% The path
	\path[draw,line width=0.6pt,postaction=decorate] (0.5, -1.5) -- (0.5, 1.5) -- (0.5, 1.5) arc (90:270:1.5);
	
	% The labels

	\node[below right] {$0$};

	\draw[xshift=0cm] (0,0) node[circle,fill,inner sep=1.2pt](a){};
	\draw[xshift=0cm] (0.5,0) node[circle,fill,inner sep=1.2pt](a){};
	\draw[xshift=0cm] (0.5,1.5) node[circle,fill,inner sep=1.2pt](a){};
	\draw[xshift=0cm] (0.5,-1.5) node[circle,fill,inner sep=1.2pt](a){};
	\draw[xshift=0cm] (-1,0) node[circle,fill,inner sep=1.2pt](a){};

	\node at (-1.4, 1) {$C(T)$};
	\node at (1.2, -0.8) {$S(T)$};
	
	\node at (0.8, 0.2) {$c$};
	\node at (1.2, 1.5) {$c + iT$};
	\node at (1.2, -1.5) {$c - iT$};
	\node at (-1.3, -0.3) {$T$};
\end{tikzpicture}
\end{minipage}
\caption{Contour $\Gamma(T)$}
\label{fig:ContourGamma}
\end{figure}
	First note that since $\abs{a^s} = a^c$, the integral converges. Suppose that $1 \leq a$, and write $a = e^\beta$ with $\beta = \log(a) \geq 0$. Let
\begin{equation*}
	f(s) = \frac{a^s}{s(s + 1)} = \frac{e^{s \beta}}{s(s + 1)}.
\end{equation*}
	Then $\res(f(s), 0) = 1$ and $\res(f(s), 1) = -\frac{1}{a}$. For $T > 0$, consider the path $\Gamma(T)$ shown in Figure. Now, we choose $T$ large enough that $0$ and $-1$ are contained in the interior of the contour $\Gamma(T)$. By the residue formula we get
\begin{equation*}
	\frac{1}{2 \pi i} \int _{\Gamma(T)} f(s) ds = 1 - \frac{1}{a}.
\end{equation*}
	Since
\begin{equation*}
	\int _{\Gamma(T)} f(s) ds = \int _{S(T)} f(s) ds + \int _{C(T)} f(s) ds,
\end{equation*}
	it suffices to show that the integral over $C(T)$ goes to $0$ as $T \to \infty$. Note that if $s \in C(T)$, then for all large $T$ we have
\begin{equation*}
	\abs{s(s + 1)} \geq \frac{1}{2} T^2.
\end{equation*}
	and since $\sigma \leq c$ we also have the estimate $\abs{e^{\beta s}} \leq e^{\beta c}$. Therefore
\begin{equation*}
	\lim _{T \to \infty} \abs{\int _{C(T)} f(s) ds} \leq \frac{C}{T^2} 2 \pi T^2 = 0
\end{equation*}
	and the case where a $\geq$ is proved. If $0 < a \leq 1$ we consider the half circle lying to the right of $S(T)$. Noting there are no poles in the interior of the contour we can give a similar argument that the integral over the half circle vanishes for $T \to \infty$.
\end{proof}


\begin{lemma}
	For all $c > 1$ we have
\begin{equation*}
	\psi_1(x) = \frac{1}{2 \pi i} \int _{c - i\infty} ^{c + i \infty} \frac{x^{s + 1}}{s(s + 1)} \cbr{-\frac{\zeta'(s)}{\zeta(s)}} ds.
\end{equation*}
\end{lemma}
\begin{proof}
	We first observe that
\begin{equation*}
	\psi(u) = \sum _{n=1} ^\infty \Lambda(n) f_n(u),
\end{equation*}
	where
\begin{equation*}
	f_n(u) = 
		\left\{
    		\begin{array}{ll}
        		1, &\text{if } n \leq u,\\
        		0, &\text{otherwise}.
        	\end{array}
		\right.
\end{equation*}
	Therefore
\begin{equation*}
\begin{aligned}	
	\psi_1(x)
		&= \int _0 ^x \psi(u) du = \sum _{n = 1} ^\infty \int _0 ^x \Lambda(n) f_n(u) \\ 
		&= \sum _{n \leq x} \Lambda(n) \int _n ^x du = \sum _{n \leq x} \Lambda(n)(x - n).
\end{aligned}
\end{equation*}
	This fact, together with previous Lemma where $a = \frac{x}{n}$, gives
\begin{equation*}
\begin{aligned}	
	\frac{1}{2 \pi i} \int _{c - i\infty} ^{c + i \infty} \frac{x^(s + 1)}{s(s + 1)} \cbr{-\frac{\zeta'(s)}{\zeta(s)}} ds 
		&= x \sum _{n = 1} ^\infty \Lambda(n) \frac{1}{2 \pi} \int _{c - i \infty} ^{c + i \infty} \frac{\cbr{\frac{x}{n}}^s}{s(s + 1)} ds \\
		&= x \sum _{n = 1} ^\infty \Lambda(n) \cbr{1 - \frac{n}{x}} \\
		&= \psi_1(x),
\end{aligned}
\end{equation*}
	as was to be shown.
\end{proof}


\begin{theorem}[Prime number theorem]
	We have that $\psi_1(x) \sim \frac{x^2}{2}$ as $x \to \infty$.
\end{theorem}
\begin{proof}
\begin{figure}[!htb]
\begin{minipage}[c]{0.5\textwidth}
\centering
\begin{equation*}
\begin{aligned}
	&\begin{aligned}
		C(T) \colon \bbr{0, \pi} &\to \field{C}, \\ \phi &\mapsto c + Te^{i (\phi + \frac{\pi}{2}) },
	\end{aligned} \\
	&\begin{aligned}
		S(T) \colon \bbr{-T, T} &\to \field{C}, \\ t &\mapsto c + it
	\end{aligned}
\end{aligned}
\end{equation*}
\end{minipage}
\begin{minipage}[c]{0.5\textwidth}
\raggedleft
\begin{tikzpicture}
	[decoration={
		markings,
		mark=at position 0.5cm with {\arrow[line width=1pt]{>}},
		mark=at position 3.5cm with {\arrow[line width=1pt]{>}}
	}]

	% The path
	\path[draw,line width=0.6pt,postaction=decorate] (-2.5, -2) -- (-2.5, 2);
	\path[draw,line width=0.6pt,postaction=decorate] (-0.5, -2) -- (-0.5, -1) -- (0.5, -1) -- (0.5, 1) -- (-0.5, 1) -- (-0.5, 2);
	\path[draw,line width=0.6pt,postaction=decorate] (2.5, -2) -- (2.5, -1) -- (1.8, -1) -- (1.8, 1) -- (2.5, 1) -- (2.5, 2);
	
	% The labels
	\node[below] at (-3.3, 0) {$s = 1$};
	\node[below] at (-0.5, 0) {$s = 1$};
	\node[below] at (2.5, 0) {$s = 1$};
	
	\node[below] at (-2.5, -2.62) {$\sigma = c$};
	\node[below] at (-0.5, -2.5) {$\gamma(T)$};
	\node[below] at (2.5, -2.5) {$\gamma(T, \delta)$};
	
	\node[right] at (2.5, -1.5) {$\gamma_1$};
	\node[below] at (2.1, -1) {$\gamma_2$};
	\node[left] at (1.8, 0) {$\gamma_3$};
	\node[above] at (2.1, 1) {$\gamma_4$};
	\node[right] at (2.5, 1.5) {$\gamma_5$};
	
	\draw[xshift=0cm] (-3.3,0) node[circle,fill,inner sep=1.2pt](a){};	
	\draw[xshift=0cm] (-0.5,0) node[circle,fill,inner sep=1.2pt](a){};
	\draw[xshift=0cm] (2.5,0) node[circle,fill,inner sep=1.2pt](a){};

	

\end{tikzpicture}
\end{minipage}
\caption{Contour $\Gamma(T)$}
\label{fig:ContourGamma}
\end{figure}
	We first fix $c > 1$, say $c = 2$ and assume that $x$ is also fixed with $x \geq 2$. Let $F(s)$ denote the integrand
\begin{equation*}
	F(s) = \frac{x^{s + 1}}{s(s + 1)} \cbr{-\frac{\zeta'(s)}{\zeta(s)}}.
\end{equation*}
	Now we deform the vertical line from $c - i \infty$ to $c + i \infty$ to the path shown in Figure. Cauchy's theorem allows us to see that
\begin{equation*}
	\frac{1}{2 \pi i} \int _{c - i \infty} ^{c + i \infty} F(s) ds = \frac{1}{2 \pi i} \int _{\gamma(T)} F(s) ds.
\end{equation*}
	Indeed we know on the basis of Proposition that $\abs{\frac{\zeta'(s)}{\zeta(s)}} \leq A\abs{t}^\eta$ for any fixed $\eta > 0$ whenever $\sigma \geq 1$ and $\abs{t} \geq 1$. Thus $\abs(F(s)) \leq A'\abs{t}^{-2 + \eta}$ in the two rectangles bounded by the line $(c - i \infty, c + i\infty)$ and $\gamma(T)$, and its decrease at infinity is rapid enough, the assertion is established.
	Next we pass from the contour $\gamma(T)$ to the contour $\gamma(T, \delta)$. For fixed T, we chose $\delta > 0$ small enough so that $\zeta$ has no zeros in the box
\begin{equation*}
	\fbr{s \in \field{C} \colon 1 - \delta \leq \sigma \leq 1, \abs{t} \leq T}.
\end{equation*}
	Such a choice can be made since $\zeta$ does not vanish on the line $\sigma = 1$. Now $F(s)$ has a simple pole at $s = 1$. In fact, by Corollary, we know that $\zeta(s) = \frac{1}{s - 1} + H(s)$, where $H(s)$ is regular near $s = 1$. Hence $-\frac{\zeta'(s)}{\zeta(s)} =\frac{1}{s - 1} + h(s)$, where $h(s)$ is holomorphic near $s = 1$, and so we have $\res(F(s), 1) = \frac{x^2}{2}$. As a result
\begin{equation*}
	\frac{1}{2 \pi i} \int _{\gamma(T)} F(s) ds = \frac{x^2}{2} + \frac{1}{2 \pi i} \int _{\gamma(T, \delta)} \frac{x^{s + 1}}{s(s + 1)} F(s) ds.
\end{equation*}
	We now the contour $\gamma(T, \delta)$ as $\gamma_1, \gamma_2, \gamma_3, \gamma_4, \gamma_5$ and estimate each of the integrals over $\gamma_j, j=1,2,3,4,5$. First we contend that there exists $T$ so large that
\begin{equation*}
	\abs{\int _{\gamma_1} F(s) ds}\leq \frac{\eps}{2} x^2
\end{equation*}
	and
\begin{equation*}
	\abs{\int _{\gamma_5} F(s) ds}\leq \frac{\eps}{2} x^2.
\end{equation*}
	To see this, we first note that for $s \in \gamma_1$ one has
\begin{equation*}
	\abs{x^{1 + s}} = x^{1 + \sigma} = x^2.
\end{equation*}
	Then by proposition we have, for example, that $\abs{\frac{\zeta'(s)}{\zeta(s)}} \leq A \abs{t}^{\frac{1}{2}}$, so
\begin{equation*}
	\abs{\int _{\gamma_1} F(s) ds} \leq C x^2 \int _T ^\infty \frac{\abs{t}\frac{1}{2}}{t^2} dt.
\end{equation*}
	Since the integral converges, we can make the right-hand side $\leq \eps \frac{x^2}{2}$ upon taking $T$ sufficiently large.The argument for the integral over $\gamma_5$ is the same. Having now fixed $T$, we choose $\delta$ appropriately small. On $\gamma_3$, note that
\begin{equation*}
	\abs{x^{1 + s}} = x^{1 + 1 - \delta} = x^{2 - \delta},
\end{equation*}
	from which we conclude that there exists a constant $C_T$ (dependent on such $T$)such that
\begin{equation*}
	\abs{\int _{\gamma_3} F(s) ds} \leq C_T x^{2 - \delta}.
\end{equation*}
	Finally, on the small horizontal segment $\gamma_2$ (and similarly on $\gamma_4$), we can estimate the integral as follows:
\begin{equation*}
	\abs{\int _{\gamma_2} F(s) ds} \leq C'_T \int _{1 - \delta} ^1 x^{1 + \sigma} d\sigma \leq C'_T \frac{x^2}{\log(x)}.
\end{equation*}
	We conclude that there exist constants $C_T$ and $C'_T$ (possibly different from the others above) such that
\begin{equation*}
	\abs{\psi_1(x) - \frac{x^2}{2}} \leq \eps x^2 + C_T x^{2 - \delta} + C'_T \frac{x^2}{\log{x}}.
\end{equation*}
	Dividing through by $\frac{x^2}{2}$, we see that
\begin{equation*}
	\abs{\frac{2 \psi_1(x)}{x^2} - 1} \leq 2\eps + 2 C_T x^{-\delta} + 2 C'_T \frac{1}{\log(x)},
\end{equation*}
	and therefore, for all large $x$ we have
\begin{equation*}
	\abs{\frac{2 \psi_1(x)}{x^2} - 1} \leq 4 \eps.
\end{equation*}
	This concludes the proof that $\psi_1(x) \sim \frac{x^2}{2}$ as $x \to \infty$.
\end{proof}

\chapter{The Critical Line}
In this chapter we will introduce the theta function, define the xi function and prove some important identities in order to come up with an integral formula that allows us to draw a far-reaching conclusion about the occurrence of zeros of the Riemann zeta function on the critical line.


\section{The Theta Function}


\begin{definition}
	The function
\begin{equation*}
\begin{aligned}
	&\vartheta \colon \field{R}_{> 0} \to \field{C}, \\
	&\vartheta(x) = \sum _{n = -\infty} ^{\infty} e^{-\pi n^2 x}
\end{aligned}
\end{equation*}
	is known as a version of the renowned "theta" function. For further references we also introduce two auxiliary functions
\begin{equation*}
	\omega(x) = \frac{1}{2}\cbr{\vartheta(x) - 1} = \sum _{n = 1} ^{\infty} e^{-\pi n^2 x}
\end{equation*}
	and
\begin{equation*}
	\theta(x) = \omega(x) - \frac{1}{2 \sqrt{t}} = \sum _{n = 1} ^{\infty} e^{-\pi n^2 x} - \frac{1}{2 \sqrt{t}}.
\end{equation*}
\end{definition}


\begin{remark}
	It is known that the well-behaved monotonically increasing function $\theta$ is defined for $0 \leq x < \infty$.
\end{remark}


\begin{proposition}
	If $f(x) = e^{-\pi x^2}$ then $f(x) = \hat{f}(\xi)$, where $\hat{f}$ is the Fourier transform of $f$, or more formally
\begin{equation}\label{equ:FourierSame}
		\int _{-\infty} ^\infty e^{-\pi x^2} e^{-2 \pi i x \xi} dx = e^{-\pi \xi^2}.
\end{equation}
\end{proposition}
\begin{proof}
\begin{figure}[!htb]
\begin{minipage}[c]{0.5\textwidth}
\centering
\begin{equation*}
\begin{aligned}
	&\begin{aligned}
		\gamma_{1_+} \colon \bbr{-R, R} &\to \field{C}, \\ r &\mapsto r,
	\end{aligned} \\
	&\begin{aligned}
		\gamma_{1_-} \colon \bbr{-R, R} &\to \field{C}, \\ r &\mapsto -r + i\xi,
	\end{aligned} \\
	&\begin{aligned}
		\gamma_{2_+} \colon \bbr{0, \xi} &\to \field{C}, \\ t &\mapsto R + it,
	\end{aligned} \\
	&\begin{aligned}
		\gamma_{2_-} \colon \bbr{0, \xi} &\to \field{C}, \\ t &\mapsto -R + i(\xi - t)
	\end{aligned}
\end{aligned}
\end{equation*}
\end{minipage}
\begin{minipage}[c]{0.5\textwidth}
\raggedleft
\begin{tikzpicture}
	[decoration={
		markings,
		mark=at position 1cm with {\arrow[line width=1pt]{>}},
		mark=at position 3cm with {\arrow[line width=1pt]{>}},
		mark=at position 4.5cm with {\arrow[line width=1pt]{>}},
		mark=at position 6cm with {\arrow[line width=1pt]{>}},
		mark=at position 8cm with {\arrow[line width=1pt]{>}},
		mark=at position 9.5cm with {\arrow[line width=1pt]{>}},
	}]
	% The axes
	\draw[help lines, very thin, ->] (-3,0) -- (3,0) coordinate (xaxis);
	\draw[help lines, very thin, ->] (0,-2) -- (0,2) coordinate (yaxis);
	\node[below, gray] at (xaxis) {$\Re(z)$};
	\node[left, gray] at (yaxis) {$\Im(z)$};

	% The path
	\path[draw,line width=0.6pt,postaction=decorate] (-2, 0) -- (2, 0) -- (2, 1) -- (-2, 1) -- (-2, 0);
	
	
	% The labels

	\node[below right] {$0$};

	\draw[xshift=0cm] (0,0) node[circle,fill,inner sep=1.2pt](a){};
	\draw[xshift=0cm] (-2,0) node[circle,fill,inner sep=1.2pt](a){};
	\draw[xshift=0cm] (2,0) node[circle,fill,inner sep=1.2pt](a){};
	\draw[xshift=0cm] (2,1) node[circle,fill,inner sep=1.2pt](a){};
	\draw[xshift=0cm] (-2,1) node[circle,fill,inner sep=1.2pt](a){};

	\node at (-0.5, -0.4) {$\gamma_{1_+}$};
	\node at (2.5, 0.5) {$\gamma_{2_+}$};
	\node at (0.5, 1.3) {$\gamma_{1_-}$};
	\node at (-2.5, 0.5) {$\gamma_{2_-}$};
	
	\node at (-2.3, -0.4) {$-R$};
	\node at (2.1, -0.4) {$R$};
	\node at (2.5, 1.4) {$R + i\xi$};
	\node at (-2.6, 1.4) {$-R + i\xi$};
\end{tikzpicture}
\end{minipage}
\caption{Contour $\gamma$}
\label{fig:ContourC}
\end{figure}
	If $\xi = 0$ we already know very well that
\begin{equation*}
	\int _{-\infty} ^\infty e^{-\pi x^2} dx = 1.
\end{equation*}
	Now, let us suppose $\xi > 0$ and consider the function $f(z) = e^{-\pi z^2}$ for $z \in \field{C}$. $f$ is clearly entire. According to Cauchy's theorem we have
\begin{equation*}
	\int _{\gamma} f(z) dz = 0.
\end{equation*}
	For the integral over $\gamma_{1_+}$ we simply have
\begin{equation*}
	\lim _{R \to \infty} \int _{-R} ^R e^{-\pi x^2} dx = 1.
\end{equation*}
	For the integral over $\gamma_{2_+}$ for a fixed $\xi$ we have
\begin{equation*}
\begin{aligned}	
	\lim _{R \to \infty} I(R) 
	&= \lim _{R \to \infty} \int _0 ^\xi f(R + i y) i dy \\
	&= \lim _{R \to \infty} \int _0 ^\xi e^{-\pi (R^2 + 2iRy - y^2)} i dy = 0,
\end{aligned}
\end{equation*}
	since we may estimate it by
\begin{equation*}
	\abs{I(R)} \leq Ce^{-\pi R^2}.
\end{equation*}
	The integral for $\gamma_{2_-}$ also goes to zero for the same reasons. Finally, the remaining integral over $\gamma_{1_-}$ is
\begin{equation*}
	\int _{R} ^{-R} e^{-\pi (x + i \xi)^2} dx = -e^{\pi \xi^2} \int _{-R} ^R e^{-\pi x^2} e^{-2 \pi i x \xi} dx.
\end{equation*}
	Thus for $R \to \infty$ the limit gives
\begin{equation*}
	1 - e^{\pi \xi^2} \int _{-\infty} ^{\infty} e^{-\pi x^2} e^{-2 \pi i x \xi} dx = 0
\end{equation*}
	and our desired formula is established. In the case $\xi < 0$, we then consider the rectangle in the lower half plane.
\end{proof}


\begin{lemma}
	For $x > 0$ we have the identity
\begin{equation*}
	\vartheta(x) = \frac{1}{\sqrt{x}} \vartheta(\frac{1}{x}).
\end{equation*}
\end{lemma}
\begin{proof}
	The change of variables $x \to \sqrt{y} x$ in the integral formula~(\ref{equ:FourierSame}) shows that the Fourier transform of the function $f(x) = e^{-\pi y x^2}$ is in fact $\hat{f}(\xi) = \frac{1}{\sqrt{y}} e^{-\frac{\pi \xi^2}{y}}$. Apparently $f(x)$ belongs to $\mathcal{F}_a$. Thus we may apply the Poisson summation formula to the pair $\cbr{f, \hat{f}}$ and we get
\begin{equation*}
	\vartheta(x) = \sum _{n = -\infty} ^\infty e^{-\pi x n^2} = \sum _{n = -\infty} ^\infty \frac{1}{\sqrt{x}} e^{-\frac{\pi \xi^2}{x}} = \frac{1}{\sqrt{x}} \vartheta(\frac{1}{x}).
\end{equation*}
\end{proof}


\begin{corollary}\label{cor:ThetaIdentity}
	We have the identities
\begin{equation*}
	\omega(x) = \frac{1}{\sqrt{x}} \omega\cbr{\frac{1}{x}} + \frac{1}{2\sqrt{x}} - \frac{1}{2}.
\end{equation*}
	and
\begin{equation*}
	\theta(x) = \frac{1}{\sqrt{x}} \theta(\frac{1}{x}).
\end{equation*}
\end{corollary}
\begin{proof}
	For the first identity we have
\begin{equation*}
	2 \omega(x) + 1 = \vartheta(x) = \frac{1}{\sqrt{x}} \vartheta(\frac{1}{x}) = \frac{1}{\sqrt{x}} \cbr{2 \omega(\frac{1}{x}) + 1}
\end{equation*}
	and just solve the equation for $\omega(x)$. The second identity follows directly from the first by writing
\begin{equation*}
\begin{aligned}	
	\theta(x) = \omega(x) - \frac{1}{2 \sqrt{x}} 
	&= \frac{1}{\sqrt{x}} \omega\cbr{\frac{1}{x}} - \frac{1}{2} \\ 
	&= \frac{1}{\sqrt{x}} \cbr{\omega\cbr{\frac{1}{x}} - \frac{\sqrt{x}}{2}} = \frac{1}{\sqrt{x}} \theta(x).
\end{aligned}
\end{equation*}
\end{proof}


\section{The Xi Function}


\begin{definition}\label{def:XiDefinition}
	Let
\begin{equation*}
\begin{aligned}
	&\xi \colon \field{C} \to \field{C}, \\
	&\xi(s) = \frac{1}{2}s(s - 1) \pi ^{-\frac{1}{2} s} \Gamma(\frac{s}{2}) \zeta(s)
\end{aligned}
\end{equation*}
	be the xi function and for further reference we define
\begin{equation*}
	\xi^*(s) = \frac{2}{s(s - 1)} \xi(s) = \pi ^{-\frac{1}{2} s} \Gamma(\frac{s}{2}) \zeta(s)
\end{equation*}
\end{definition}


\begin{remark}
	$\xi$ is entire since $s$ and $(s - 1)$ cancel out the simple poles of $\Gamma$ in $s = 0$ and $\zeta$ in $s = 1$ and since the trivial zeros of $\zeta$ eradicate the simple poles of $\Gamma$ in $s = -2n$ for $n \in \field{N}_{> 0}$.
\end{remark}


\begin{proposition}\label{pro:XiIdentity}
	We have the identity
\begin{equation*}
	\xi(s) = \xi(1 - s).
\end{equation*}
\end{proposition}
\begin{proof}
	Let us recall the formula
\begin{equation*}
	\Gamma(\frac{1}{2}s) = \int _0 ^\infty e^{-y} y^{\frac{1}{2}s - 1} dy,
\end{equation*}	
	whenever $\sigma > 0$. We observe that if $n \geq 1$ and by substituting $y = x \pi n^2$ we may write
\begin{equation*}
	\int _0 ^\infty e^{-n^2 \pi x} x^{\frac{1}{2} s - 1} dx = n^{-s} \pi^{-\frac{1}{2}s}\Gamma(\frac{1}{2}s).
\end{equation*}
	Hence if $\sigma > 1$ we get
\begin{equation*}
\begin{aligned}
	\xi^*(s) = \pi^{-\frac{1}{2}s} \Gamma(\frac{1}{2}s) \zeta(s) 
		&= \sum _{n = 1} ^\infty \int _0 ^\infty x^{\frac{1}{2} s - 1} e^{-n^2 \pi x} dx \\
		&= \int _0 ^\infty x^{\frac{1}{2} s - 1} \sum _{n = 1} ^\infty e^{-n^2 \pi x} dx
		= \int _0 ^\infty x^{\frac{1}{2} s - 1} \omega(x) dx,
\end{aligned}
\end{equation*}
	where we can interchange sum and integral because of absolute convergence. We now use Corollary~\ref{cor:ThetaIdentity} and may now write
\begin{equation*}
\begin{aligned}	
	\xi^*(s) &= \\
		\int _0 ^1 x^{\frac{1}{2} s - 1} \omega(x) dx  + \int _1 ^\infty x^{\frac{1}{2} s - 1} \omega(x) dx &= \\
		\int _0 ^1 x^{\frac{1}{2} s - 1} \cbr{\frac{1}{\sqrt{x}} \omega\cbr{\frac{1}{x}} + \frac{1}{2\sqrt{x}} - \frac{1}{2}} dx  + \int _1 ^\infty x^{\frac{1}{2} s - 1} \omega(x) dx &= \\
		\frac{1}{s - 1} - \frac{1}{s} + \int _0 ^1 x^{\frac{1}{2} s - \frac{3}{2}} \omega(x) dx + \int _1 ^\infty x^{\frac{1}{2} s - 1} \omega(x) dx &= \\
		\frac{1}{s - 1} - \frac{1}{s} + \int _1 ^\infty \cbr{x^{-\frac{1}{2} s - \frac{1}{2}} + x^{\frac{1}{2} s - 1}} \omega(x) dx &.
\end{aligned}
\end{equation*}
	Since the function $\omega$ has exponential decay at infinity the last integral defines an entire function. Thus the formula holds by analytic continuation for all $s \in \field{C}$ with simple poles at $s = 0$ and $s = 1$. The right-hand side remains unchanged when $s$ is replaced by $1 - s$. Hence we finally get
\begin{equation*}
	\xi(s) = \frac{1}{2}s(1 - s) \xi^*(s) = \frac{1}{2}(1 - s)s \xi^*(1 - s) = \xi(1 - s).
\end{equation*}
\end{proof}


\begin{lemma}\label{lem:XiMellin}
	Let $f(x) = 2 \theta(x^2)$. For $0 < \sigma < 1$ we have
\begin{equation*}
	\xi^*(s) = \int _0 ^\infty x^{s - 1} f(x) dx,
\end{equation*}
	hence $\xi^*(s) = \fbr{\mathcal{M} f}$ and therefore by Mellin's inversion formula $f = \fbr{\mathcal{M}^{-1} \xi^*}$ whenever $0 < \sigma < 1$.
\end{lemma}
\begin{proof}
	This formula is essentially the same as stated in \cite{Edwards1974} [p. 213] after the substitution $u \to \frac{1}{x}$.
\end{proof}


\begin{definition}
	Let
\begin{equation*}
\begin{aligned}
	&\Xi \colon \field{R} \to \field{C}, \\
	&\Xi(t) = \xi(\frac{1}{2} + it) = -\frac{1}{2}(t^2 + \frac{1}{4}) \pi ^{-\frac{1}{4} - \frac{1}{2} it} \Gamma(\frac{1}{4} + \frac{1}{2} it) \zeta(\frac{1}{2} + it)
\end{aligned}
\end{equation*}
	be the restriction of the xi function to $\sigma = \frac{1}{2}$.
\end{definition}


\begin{lemma}
	$\Xi$ is even.
\end{lemma}
\begin{proof}
	Let $t \in \field{R}$. We have
\begin{equation*}
	\Xi(-t) = \xi(\frac{1}{2} - it) = \xi(1 - (\frac{1}{2} + it)) = \xi(\frac{1}{2} + it) = \Xi(t).
\end{equation*}
\end{proof}


\begin{lemma}
	$\Xi$ is a real function.
\end{lemma}
\begin{proof}
	We first recall Corollary~\ref{cor:ThetaIdentity} which immediately yields the symmetric form
\begin{equation*}
	x^{-\frac{1}{4}} \theta(\frac{1}{x}) = x^{\frac{1}{4}} \theta(x).
\end{equation*}
	Plugging this into the integral form of the functional equation of $\xi^*$ from Lemma~\ref{lem:XiMellin} for $0 < \sigma < 1$ gives
\begin{equation*}
	\xi^*(s) = \int _0 ^\infty x^{\frac{s - \frac{1}{2}}{2}} x^{\frac{1}{4} - 1} \theta(x) dx = \int _0 ^\infty x^{-\frac{s - \frac{1}{2}}{2}} x^{\frac{1}{4} - 1} \theta(x) dx.
\end{equation*}
	By addition of the two sides we obtain
\begin{equation*}
	\xi^*(s) = \int _0 ^\infty \cosh\cbr{\frac{1}{2} \cbr{s - \frac{1}{2}} \log(x)} x^{\frac{1}{4} - 1} \theta(x) dx.
\end{equation*}
	Further, by applying the addition theorem
\begin{equation*}
	\cosh(x + y) = \cosh(x)\cosh(y) + \sinh(x) \sinh(y)
\end{equation*}	
	for all $x,y \in \field{R}$ we may deduce that
\begin{equation*}
\begin{aligned}	
	\xi^*(s) 
		&= \int _0 ^\infty \cosh\cbr{\frac{1}{2} \cbr{\sigma - \frac{1}{2}} \log(x)} \cos\cbr{\frac{t}{2} \log(x)} x^{\frac{1}{4} - 1} \theta(x) dx \\
		&+ \int _0 ^\infty \sinh\cbr{\frac{1}{2} \cbr{\sigma - \frac{1}{2}} \log(x)} \sin\cbr{\frac{t}{2} \log(x)} x^{\frac{1}{4} - 1} \theta(x) dx.
\end{aligned}
\end{equation*}
	We now in fact observe that
\begin{equation*}
	\Im(\Xi(t)) = \lim _{\sigma \to \frac{1}{2}} \Im(\xi(\sigma + it)) = \lim _{\sigma \to \frac{1}{2}} \Im\cbr{\frac{(\sigma + it)(\sigma + it - 1)}{2}\xi^*(\sigma + it)} = 0,
\end{equation*}
	hence $\Xi$ is entirely real.
\end{proof}


\section{Roots of $\zeta$ on the Critical Line}


\begin{proposition}
	We have the identity
\begin{equation*}	
	\int _0 ^{\infty} \frac{\Xi(t)}{t^2 + \frac{1}{4}} \cos(xt) dt = \frac{1}{2} \pi (e^{\frac{1}{2} x} - 2e^{-\frac{1}{2}x} \omega(e^{-2x})).
\end{equation*}
\end{proposition}
\begin{proof}
	Let $f(t) = \abs{\gamma(t)}^2 = \lambda(it) \lambda(-it)$ where $\lambda$ is analytic and let
\begin{equation*}
	\Phi(x) = \int _0 ^{\infty} f(t) \Xi(t) \cos(xt) dt.	
\end{equation*}
	By substituting $y = e^x$ we get
\begin{equation*}
\begin{aligned}
	\Phi(x) 
		&= \int _0 ^\infty \lambda(it) \lambda(-it) E(t) \frac{1}{2}(y^{it} + y^{-it}) dt \\
		&= \frac{1}{2} \int _{-\infty} ^\infty \lambda(it) \lambda(-it) E(t) y^{it} dt \\
		&= \frac{1}{2} \int _{-\infty} ^\infty \lambda(it) \lambda(-it) \xi(\frac{1}{2} + it) y^{it} dt
\end{aligned}
\end{equation*}
	Again the substitution $s = \frac{1}{2} + it$ gives
\begin{equation*}
\begin{aligned}
	\Phi(x) 
		&= \frac{1}{2 i\sqrt{y}} \int _{\frac{1}{2} -\infty} ^{\frac{1}{2} + \infty} \lambda(s - \frac{1}{2}) \lambda(\frac{1}{2} - s) \xi(s) y^{s} dt \\
		&= \frac{1}{2 i\sqrt{y}} \int _{\frac{1}{2} -\infty} ^{\frac{1}{2} + \infty} \lambda(s - \frac{1}{2}) \lambda(\frac{1}{2} - s) (s - 1) \Gamma(1 + \frac{1}{2}s) \pi ^{-\frac{1}{2} s} \zeta(s) y^{s} ds.
\end{aligned}
\end{equation*}
	Finally setting $\lambda(s) = \frac{1}{s + \frac{1}{2}}$ together with Lemma~\ref{lem:XiMellin} yields 
\begin{equation*}
\begin{aligned}
	\Phi(x) 
		&= -\frac{1}{2 i\sqrt{y}} \int _{\frac{1}{2} -\infty} ^{\frac{1}{2} + \infty} \frac{1}{s} \Gamma(1 + \frac{1}{2} s) \pi ^{-\frac{1}{2} s} \zeta(s) y^{s} ds \\
		&= -\frac{1}{4 i\sqrt{y}} \int _{\frac{1}{2} -\infty} ^{\frac{1}{2} + \infty} \Gamma(\frac{1}{2} s) \pi ^{-\frac{1}{2} s} \zeta(s) y^{s} ds \\
		&= -\frac{\pi}{2 \sqrt{y}} \cbr{\frac{1}{2 i \pi} \int _{\frac{1}{2} -\infty} ^{\frac{1}{2} + \infty} \xi^*(s) \cbr{\frac{1}{y}}^{-s} ds} \\
		&= -\frac{\pi}{\sqrt{y}} \omega\cbr{\frac{1}{y^2}} + \frac{1}{2} \pi \sqrt{y}.
\end{aligned}
\end{equation*}
	After re-substituting we receive the identity we wanted to show.
\end{proof}


\begin{theorem}
	$\zeta$ has infinitely many roots in $s = \frac{1}{2} + t$ for $t \in \field{R}$.
\end{theorem}
\begin{proof}
	We substitute $x = -i \alpha$ and plug it into the previously shown identity
\begin{equation*}
\begin{aligned}
	\frac{2}{\pi} \int _0 ^{\infty} \frac{\Xi(t)}{t^2 + \frac{1}{4}} \cosh(\alpha t) dt 
	&= e^{-\frac{1}{2} \alpha} - 2e^{\frac{1}{2} i \alpha} \omega(e^{2 i \alpha}) \\
	&= 2 \cos(\frac{1}{2} \alpha) - 2e^{\frac{1}{2} i \alpha} \cbr{\frac{1}{2} + \omega(e^{2 i \alpha})}.
\end{aligned}
\end{equation*}
	Since $\zeta(\frac{1}{2} + it) = \mathcal{O}(t^A)$ and $\Xi(t) = \mathcal{O}(t^A e^{-\frac{1}{4} \pi t})$ and since $\cosh(\alpha t) = \mathcal{O}(e^{\alpha t})$, for $\alpha < \frac{1}{4} \pi$ we can interchange integral and differentiation according to the dominated convergence theorem and differentiate both sides with respect to $\alpha$ on both sides $2n$-times and receive
\begin{equation*}
\begin{aligned}	
	\frac{2}{\pi} \int _0 ^{\infty} \frac{\Xi(t)}{t^2 + \frac{1}{4}} t^{2n}
		&\cosh(\alpha t) dt \\
		&= \frac{(-1)^{n} \cos(\frac{1}{2} \alpha)}{2^{2n - 1}} - 2 \cbr{\frac{d}{d\alpha}}^{2n}e^{\frac{1}{2} i \alpha} \cbr{\frac{1}{2} + \omega(e^{2 i \alpha})}.
\end{aligned}
\end{equation*}
	In order to show that the last term vanishes as $\alpha \to 0$ for every fixed $n$ we use Corollary~\ref{cor:ThetaIdentity} and write
\begin{equation*}
\begin{aligned}
	\omega(i + \delta) = \sum _{n=1} ^{\infty} e^{-n^2 \pi (i + \delta)}
		&= \sum _{n=1} ^{\infty} (-1)^n e^{-n^2 \pi \delta} \\
		&= 2 \omega(4 \delta) - \omega(\delta) \\
		&= \frac{1}{\sqrt \delta} \omega\cbr{\frac{1}{4\delta}} - \frac{1}{\sqrt \delta} \omega\cbr{\frac{1}{\delta}} - \frac{1}{2}.
\end{aligned}
\end{equation*}
	Obviously along every route in an angle $\arg(x - i) < \frac{1}{2} \pi$ we have
\begin{equation*}
	\lim _{x \to i} \omega(x) + \frac{1}{2} = 0.
\end{equation*}
	Therefore we have shown that
\begin{equation*}
	\lim _{\alpha \to \frac{1}{4} \pi} \frac{2}{\pi} \int _0 ^{\infty} \frac{\Xi(t)}{t^2 + \frac{1}{4}} t^{2n} \cosh(\alpha t) dt = \frac{(-1)^{n} \cos(\frac{1}{8} \pi)}{2^{2n - 1}}.
\end{equation*}
	Let's assume that $\Xi$ were ultimately of one sign, say positive for $t \geq T$ where $T \in \field{R}$. Then for all $L > 0$ and $T' \geq T$ we have
\begin{equation*}
	\lim _{\alpha \to \frac{1}{4} \pi} \frac{2}{\pi} \int _T ^{\infty} \frac{\Xi(t)}{t^2 + \frac{1}{4}} t^{2n} \cosh(\alpha t) dt = L.
\end{equation*}
	Hence
\begin{equation*}
	\lim _{\alpha \to \frac{1}{4} \pi} \frac{2}{\pi} \int _T ^{T'} \frac{\Xi(t)}{t^2 + \frac{1}{4}} t^{2n} \cosh(\alpha t) dt \leq L.
\end{equation*}
	Hence for $\alpha = \frac{1}{4}\pi$ we have
\begin{equation*}
	\frac{2}{\pi} \int _T ^{T'} \frac{\Xi(t)}{t^2 + \frac{1}{4}} t^{2n} \cosh(\frac{1}{4}\pi t) dt \leq L.
\end{equation*}
	Thus the integral
\begin{equation*}
	\frac{2}{\pi} \int _0 ^{\infty} \frac{\Xi(t)}{t^2 + \frac{1}{4}} t^{2n} \cosh(\frac{1}{4}\pi t) dt
\end{equation*}
	is convergent and therefore the integral is uniformly convergent with respect to $0 \leq \alpha \leq \frac{1}{4}\pi$ and it follows that
\begin{equation*}
	\int _0 ^{\infty} \frac{\Xi(t)}{t^2 + \frac{1}{4}} t^{2n} \cosh(\frac{1}{4} \pi t) dt = \frac{(-1)^{n} \pi \cos(\frac{1}{8} \pi)}{2^{2n}}.
\end{equation*}
	Now we choose $n$ to be odd. The right-hand side is negative and therefore
\begin{equation*}
\begin{aligned}
	\int _T ^{\infty} \frac{\Xi(t)}{t^2 + \frac{1}{4}} t^{2n} \cosh(\frac{1}{4} \pi t) dt 
		&< - \int _0 ^{T} \frac{\Xi(t)}{t^2 + \frac{1}{4}} t^{2n} \cosh(\frac{1}{4} \pi t) dt \\
		&< KT^{2n},
\end{aligned}
\end{equation*}
	where $K$ is independent of $n$. But by hypothesis there is a $m = m(T)$ such that
\begin{equation*}
	\frac{\Xi(t)}{t^2 + \frac{1}{4}} \geq m
\end{equation*}
	for $2T \leq t \leq 2T + 1$. Hence
\begin{equation*}
	\int _T ^{\infty} \frac{\Xi(t)}{t^2 + \frac{1}{4}} t^{2n} \cosh(\frac{1}{4} \pi t) dt \geq \int _{2T} ^{2T + 1} mt^{2n} dt \geq m(2T)^{2n}.
\end{equation*}
	Hence
\begin{equation*}
	m2^{2n} < K,
\end{equation*}
	which is false for sufficiently large $n$. That proves our theorem by contradiction.
\end{proof}

\appendix
\bibliographystyle{Style}
\bibliography{References}
\nocite{*}

\end{document}